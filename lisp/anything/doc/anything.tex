\documentclass[a4paper,11pt]{article}
\usepackage[french]{babel}
\usepackage[utf8]{inputenc}
\usepackage{textcomp}
\usepackage{graphicx}
\usepackage{array}
\usepackage[pdftex=true,
           hyperindex=true,
           colorlinks=true]{hyperref}
\usepackage{url}
\usepackage{natbib}
\usepackage{setspace}


\begin{document}
\begin{titlepage}
\title{Anything User Manual}
\date{Last update: \today}
\author{ThierryVolpiatto}
\maketitle
\tableofcontents
\end{titlepage}

\section{Install}
\label{sec:install}

First get the files from git repo:

Anything git repo is at:\\
\url{http://repo.or.cz/w/anything-config.git}

To get it with git:

\begin{verbatim}
git clone git://repo.or.cz/anything-config.git
\end{verbatim}

NOTE: Files are published on Emacswiki, but be aware that it is \underline{unsafe} to get files from Emacswiki.
          
You need 2 files:
\begin{itemize}
\item anything.el
\end{itemize}
Contain the anything engine.

\begin{itemize}
\item anything-config.el
\end{itemize}
Contain all the sources and preconfigured functions ready to use.

and optionally:
\begin{itemize}
\item anything-match-plugin.el
\end{itemize}
Allow matching multi pattern search when entering a space in prompt.

Once downloaded these files, put them in your `load-path'.
If you don't know what is your load-path do C-h v load-path.

\section{Config}
\label{sec:config}
Edit your ~/.emacs.el file and add:

\begin{verbatim}
(require 'anything-config)
\end{verbatim}

Optionally, you can add:

\begin{verbatim}
(require 'anything-match-plugin)
\end{verbatim}

Note that if you don't require anything-match-plugin, you can enable/disable it afterward with:

M-x anything-c-toggle-match-plugin

NOTE:
It is not recommended to use the variable `anything-sources', please use instead the preconfigured anything command
you will find in anything-config.el or build your own.

Be aware also that making your own anything commands with a lot of sources involved can be very costly and slowdown
anything a lot.

\section{General anything commands}
\label{sec:gener-anyth-comm}
Anything allow you to have few binding to remember unlike all others Emacs applications.

Anything show you by default in mode-line the most useful bindings, you will see in headers of anything buffer some more specific
commands.

So when anything start what you have to remember:

\begin{itemize}
\item Access to action menu with 
\begin{verbatim}
TAB
\end{verbatim}
\end{itemize}
\begin{itemize}
\item Use persistent actions with
\begin{verbatim}
C-z
\end{verbatim}
\end{itemize}
\begin{itemize}
\item Mark candidate with
\begin{verbatim}
M-<SPACE>
\end{verbatim}
\end{itemize}

So three bindings to remember and they are anyway documented in mode-line.
For more, hitting
\begin{verbatim}
C-h m
\end{verbatim}
while in anything session will show you all other bindings.

\subsection{All anything bindings}
\label{sec:all-anyth-bind}

\begin{tabular}{| l | m{5cm} |}
  \hline
  Command & Key \\
  \hline
  anything-next-line & down or C-n \\
  \hline
  anything-previous-line & up or C-p \\
  \hline
  anything-previous-page & prior or M-v \\
  \hline
  anything-next-page & next or C-v \\
  \hline
  anything-beginning-of-buffer & 
\begin{verbatim}
M-<
\end{verbatim}
  \\
  \hline
  anything-end-of-buffer &
\begin{verbatim}
M->
\end{verbatim}
  \\
  \hline
  anything-next-source & right or C-o \\
  \hline
  anything-previous-source & left \\
  \hline
  anything-exit-minibuffer & RET \\
  \hline
  anything-select-with-digit-shortcut & C-1/9/A/Z \\
  \hline
  anything-select-action & TAB or C-i \\
  \hline
  anything-execute-persistent-action & C-z \\
  \hline
  anything-select-2nd-action-or-end-of-line & C-e \\
  \hline
  anything-select-3rd-action & C-j \\
  \hline
  anything-scroll-other-window & M-next or C-M-v \\
  \hline
  anything-scroll-other-window-down & M-prior or C-M-y \\
  \hline
  anything-toggle-visible-mark & C-SPACE \\
  \hline
  anything-prev-visible-mark & M-[ \\
  \hline
  anything-next-visible-mark & M-] \\
  \hline
  anything-delete-minibuffer-contents & C-k \\
  \hline
  anything-isearch & C-s \\
  \hline
  anything-toggle-resplit-window & C-t \\
  \hline
  anything-quit-and-find-file & C-x C-f \\
  \hline
  anything-delete-current-selection & C-c C-d \\
  \hline
  anything-yank-selection & C-c C-y \\
  \hline
  anything-kill-selection-and-quit & C-c C-k \\
  \hline
  anything-follow-mode & C-c C-f \\
  \hline
  anything-force-update & C-c C-u \\
  \hline
  anything-debug-output & C-c C-x C-d \\
  \hline
  anything-display-all-visible-marks & C-c C-x C-m \\
  \hline
  anything-send-bug-report-from-anything & C-c C-x C-b \\
  \hline
  anything-previous-history-element & M-p \\
  \hline
  anything-next-history-element & M-n \\
  \hline
  anything-toggle-all-marks & M-m \\
  \hline
  anything-yank-text-at-point & C-w \\
  \hline
\end{tabular}
\section{Overview of preconfigured anything commands}
\label{sec:overv-comm-avail}
For starting with anything, a set of commands have been set for you in anything menu.
The bindings of all these commands are prefixed with `f5-a'.

To discover more anything commands run from menu anything all commands (anything-execute-anything-command).
Or run anything-M-x (f5-a M-x) and type anything.

When you like a command, e.g f5 a M-x you should bind it to something more convenient like M-x to replace the Emacs
original keybinding.

\section{Anything Find Files}
\label{sec:anything-find-files}
`anything-find-files' provide you a way to navigate in your system file easily.
All the actions you can do on files from here are described in this section.

It is binded in menu, and in `anything-command-map' to f5-a C-x C-f. \\
We will assume you have binded `anything-find-files' to C-x C-f.
To do that put in your .emacs.el:
\begin{verbatim}
(global-set-key (kbd "C-x C-f") 'anything-find-files)
\end{verbatim}
It is well integrated with tramp, you can enter any tramp filename and it will complete.
(e.g /su::, /sudo::, /ssh:host:, ... etc)

Called with a prefix arg, (C-u) anything-find-files will show you also history of last visited directories.

\subsection{Navigation}
\label{sec:navigation}
Anything-find-files is not by default on ~/ but on default-directory or
thing-at-point as it use ffap.If you are on a url, a mail adress
etc.. it will do the right thing.

So anything-find-files work like find-file (C-x C-f), but if you use it
with anything-match-plugin.el, you have to add a space and then the next
part of pattern you want to match:

Example:

\begin{verbatim}
Find Files or url: ~/
That show all ~/ directory.

Find Files or url: ~/des
will show all what begin with "des"

Find Files or url: ~/ esk
(Notice the space after ~/) will show all what contain esk.

Find Files or url: ~/ el$
Will show all what finish with el

\end{verbatim}
You can move in the anything buffer with C-n C-p or arrow keys, when you
are on a file, you can hit C-z to show only this file-name in the
anything buffer.
On a directory, C-z will switch to this directory to continue searching
in it.
On a symlink C-z will expand to the true name of symlink.(moving your
mouse cursor over a symlink will show the true name of it).

So it is quite easy to navigate in your files with anything-find-files.

Forget to mention C-. that go to root of current dir or to precedent
level of dir.
So for example you can hit C-z and then come back immediatly where you
were with C-. instead of erasing minibuffer input with DEL.
On non graphic display, it is bound to C-l.

If you like it you can safely bind it to C-x C-f to replace the standard
find-file:

\begin{verbatim}
(global-set-key (kbd "C-x C-f") 'anything-find-files)
\end{verbatim}

NOTE:
Starting anything-find-files with C-u will show you a little history of the last visited directories.


\subsection{Jump with nth commands}
\label{sec:jump-with-nth}
Take advantage of the second, third and 4th actions in anything.
Instead of opening action menu with TAB, just hit:

C-e for 2th action\\
C-j for 3th action\\

You can bind 4th action to some key like this:
\begin{verbatim}
(define-key anything-map (kbd "<C-tab>") 'anything-select-4th-action)
\end{verbatim}
\newpage
\subsection{Anything find files action shortcuts}
\label{sec:anything-find-files-1}
Instead of having to open action pannel with TAB, 
you have some convenients shortcuts to quickly run actions.
Use C-c ? from an anything-find-files session to have a description.\\

\begin{tabular}{| l | l |}
  \hline
  Command & Key \\
  \hline
  anything-ff-run-grep & M-g s (C-u recurse)\\
  \hline
  anything-ff-run-rename-file & M-R (C-u Follow)\\
  \hline
  anything-ff-run-copy-file & M-C (C-u Follow)\\
  \hline
  anything-ff-run-byte-compile-file & M-B (C-u Load)\\
  \hline
  anything-ff-run-load-file & M-L \\
  \hline
  anything-ff-run-symlink-file & M-S (C-u Follow)\\
  \hline
  anything-ff-run-delete-file & M-D \\
  \hline
  anything-ff-run-switch-to-eshell & M-e \\
  \hline
  anything-ff-run-complete-fn-at-point & M-tab \\
  \hline
  anything-ff-run-switch-other-window & C-o \\
  \hline
  anything-ff-run-open-file-externally & C-c C-x (C-u choose)\\
  \hline
  anything-ff-help & C-c ? \\
  \hline
  anything-ff-rotate-left-persistent & M-l \\
  \hline
  anything-ff-rotate-right-persistent & M-r \\
  \hline
  anything-find-files-down-one-level & C-. or C-l \\
  \hline
  anything-ff-properties-persistent & M-i \\
  \hline
\end{tabular}
\subsection{Turn in image viewer}
\label{sec:turn-image-viewer}
You can turn anything-find-files in a nice image-viewer.

Navigate to your image directory, then type C-u C-z on first image.
Now turn on `follow-mode' with C-c C-f.
You can now navigate in your image directory with arrow up and down or C-n C-p.
Don't forget also to use C-t to split you windows vertically if needed.

You will find also two actions to rotate image in action menu. 
To use these actions whitout quitting, use M-l (rotate left) and M-r (rotate right).
Of course M-l and M-r have no effect if candidate is not an image file.\\

NOTE:
It use image-dired in background, so if image-dired don't work for some reason, this will
not work too.
Be sure to have Imagemagick package installed.\\

\includegraphics[width=15cm]{image-viewer1}
\newpage

\subsection{Serial rename}
\label{sec:serial-rename}
You can rename files with a new prefix name and by incremental number.
The marked files will be renamed with a new prefix name and starting
at the start-number you have choosen.\\
Note that the marked files are in the order of the selection you did, this allow to reorder
files.\\
If you mark files in other directories than the current one, these files will be moved or symlinked to current one.\\

TIP: If you have more than 100 files to serial rename, start at 100 instead of one to have your directory
sorted correctly.\\

You have to ways to serial rename:\\
\begin{itemize}
\item By renaming: All the file of others directories are moved in directory where renaming happen.\\ 
\end{itemize}
\begin{itemize}
\item By symlinking: All the files that are not files of the directory where you want to rename will be symlinked,
others will be renamed.\\
\end{itemize}

Example of Use:\\
I want to create a directory with many symlinked images coming from various directories.\\
1) C-x C-f (launch anything-find-files)\\
2) Navigate to the place of your choice and write new directory name ending with ``/'' and press RET.\\
3) Navigate and browse images, when you want an image mark it, you can mark all in a directory with M-a.\\
4) When you have marked all files, choose ``serial rename by symlinking'' in action menu.\\
5) Choose new name and start number.\\
6) Navigate to initial directory (where files will be renamed/symlinked) and RET.\\
7) Say yes to confirm, that's done.\\
8) Start viewing your pictures.\\

\subsection{Grep}
\label{sec:grep}
We describe here anything-do-grep, an incremental grep.
It is really convenient as you can start a search just after finding the place or file(s) you want to search in.
By the nature of incremental stuff, it is faster than original Emacs grep for searching.

As you type the display change (like in all other anything commands).
This grep is also recursive unlike the emacs implementation that use find/xargs.

It support wildcard and (re)use the variables `grep-find-ignored-files'
and `grep-find-ignored-directories'.

It have full tramp integration.
(you can grep file on a remote host or in su/sudo methods).

\begin{itemize}
\item NOTE: 
You will find a file named anything-grep.el in extensions.
It is NOT needed to use with what is described here.
It is another implementation of grep but not incremental.
\end{itemize}


\begin{itemize}
\item NOTE:
When using it recursively, `grep-find-ignored-files' is not used unless you don't specify
the only extensions of files where you want to search (you will have a prompt).
You can now specify more than one extension to search.\\
e.g *.el *.py *.tex \\
will search only in files with these extensions.
\end{itemize}

\begin{itemize}
\item NOTE: Windows users need grep version 
\begin{math}
\geq2.5.4
\end{math}
of Gnuwin32 on windoze.
This version should accept the --exclude-dir option.
\end{itemize}

\newpage
Grep in action on current file: \\
\\
\includegraphics[width=15cm]{grep-screenshot2}

\subsection{Anything do grep}
\label{sec:anything-do-grep}
Start with M-x anything-do-grep bound to f5 a M-g s
A prefix arg will launch recursive grep.

\subsection{Grep from anything-find-files}
\label{sec:grep-from-anything}
From anything-find-files (f5 a C-x C-f) Open the action menu with tab and choose grep.
A prefix arg will launch recursive grep.\\

\begin{itemize}
\item NOTE:You can now launch grep with (C-u) M-g s without switching to the action pannel.
\end{itemize}

\subsection{Grep One file}
\label{sec:grep-one-file}
Just launch grep, it will search in file at point.
if file is a directory, it will search in ALL files of this directory like:

\begin{verbatim}
grep -nH -e pattern *
\end{verbatim}


\subsection{Grep Marked files}
\label{sec:grep-marked-files}
Just mark some files with
\begin{verbatim}
C-<SPACE>
\end{verbatim}
and launch grep.
\subsubsection{Grep marked files from differents directories}
\label{sec:grep-marked-files-1}
This is a very nice feature of anything grep implementation that allow to search in specific files located not
only in current directory but anywhere in your file system.

To use navigate in your file system and mark files with 
\begin{verbatim}
C-<SPACE> 
\end{verbatim}
When you have marked all files, just launch grep in action menu.

NOTE: Using prefix-arg (C-u) will start a recursive search with the extensions of the marked files
except if those are one of ``grep-find-ignored-files''.  

\subsection{Grep Directory recursively}
\label{sec:grep-direct-recurs}
From anything-find-files, reach the root of the directory where you want to search in,
then hit TAB to open the action menu and choose grep with a prefix arg (i.e C-u RET).

if you want to use grep directly from anything-do-grep, do:

\begin{verbatim}
C-u f5 a M-g s
\end{verbatim}

You will be prompted for selecting in which category of files to search:
Use the wilcard syntax like *.el for example (search in only ``.el'' files).
\\
\begin{itemize}
\item NOTE: Be sure to be at the root of directory, of course grepping recursively with cursor
on a filename candidate will find nothing.

\end{itemize}
\subsection{Grep Using Wildcard}
\label{sec:grep-using-wildcard}
You can use wildcard:
From the root of your directory, if you want for example to search files with .el extension:
add *.el to prompt.

\subsection{Grep thing at point}
\label{sec:grep-thing-at}
Before lauching anything, put your cursor on the start of symbol or sexp you will want to grep.
Then launch anything-do-grep or anything-find-files, and when in the grep prompt hit C-w as many time as needed.

\subsection{Grep persistent action}
\label{sec:grep-pers-acti}
As always, C-z will bring you in the buffer corresponding to the file you are grepping. \\
Well nothing new, but using C-u C-z will record this place in the mark-ring.
So if you want to come back later to these places no need to grep again, you will find all these 
places in the mark-ring.\\
Accessing the mark-ring in Emacs is really inconvenient, fortunately, you will find in anything-config 
``anything-all-mark-ring'' which is a mark-ring browser (f5-a C-c SPACE).
\newpage
``anything-all-mark-ring'' is in anything menu also, in the tool section.\\

\begin{itemize}
\item TIP: Bind ``anything-all-mark-ring'' to C-c SPACE.
\begin{verbatim}
(global-set-key (kbd "C-c <SPC>") 'anything-all-mark-rings)
\end{verbatim}
\end{itemize}

\begin{itemize}
\item NOTE: ``anything-all-mark-ring'' handle global-mark-ring also.
\end{itemize}

\subsection{Save grep session}
\label{sec:save-grep-session}
If you want to save the results of your grep session, doing ``C-x C-s'' will save your grep results
in a *grep* buffer. 

\subsection{Open Files Externally}
\label{sec:open-files-extern}
You will find in action menu from anything-find-files an action to open file with external program.
If you have no entry in .mailcap or /etc/mailcap, you will enter an anything session to choose a program
to use with this kind of file.
It will offer to you to save setting to always open this kind of files with this program.
Once configured, you can however open the files of same extension with some other program by forcing anything
to choose program with C-u.\\

NOTE: You can now open files externally with ``C-c C-x'' from anything-find-files.

\subsection{Eshell command on files}
\label{sec:eshell-command-files}
You can run eshell-command on files or marked files, the command you use have to accept
one file as argument.
The completion is make on your eshell aliases.
This allow you creating personal actions for ``anything-find-files''.

\subsection{Why Eshell}
\label{sec:why-eshell}
\begin{itemize}
\item Because eshell allow you to create aliases.
\end{itemize}
\begin{itemize}
\item Because eshell accept shell commands but also elisp functions.
\end{itemize}
All these command should end with \$1.
You will have completion against all these aliases once eshell is loaded.
(start it once before using anything-find-files).

\subsection{Setup Aliases}
\label{sec:setup-aliases}
Go in eshell, an enter at prompt:
\begin{verbatim}
alias my_alias command \$1
\end{verbatim}
NOTE: don't forget to escape the \$.

See the documentation of Eshell for more info.

\subsection{Problem starting Eshell}
\label{sec:probl-start-eshell}
Eshell code is available (autoloaded) only when you have started once eshell.
That's annoying like many autoloaded stuff in Emacs.
 
\subsection{Dired Commands}
\label{sec:dired-commands}
To enable some of the usual commands of dired, put in .emacs.el
\begin{verbatim}
(anything-dired-bindings 1)
\end{verbatim}
Or run interactively:
\begin{verbatim}
M-x anything-dired-bindings
\end{verbatim}
This will replace in dired C, R, S, and H commands.
That is copy, rename, symlink, hardlink.
When creating a symlink, you will find relsymlink in actions menu.(TAB).

\subsection{Copy Files}
\label{sec:copy-files}
It is a powerfull feature of anything-find-files as you can mark files in very different places in your
file system and copy them in one place.

Dired is not able to do that, you can mark files only in current dired display and copy them somewhere.

So, easy to use, just mark some files, and hit copy files in the action menu.
That will open a new anything-find-files where you can choose destination.
\subsection{Rename Files}
\label{sec:rename-files}
Just mark some files, and hit rename files in the action menu.
That will open a new anything-find-files where you can choose destination.
\subsection{Symlink Files}
\label{sec:symlink-files}
Just mark some files, and hit symlink or relsymlink files in the action menu.
That will open a new anything-find-files where you can choose destination.
\subsection{Hardlink}
\label{sec:hardlink}
Just mark some files, and hit hardlink files in the action menu.
That will open a new anything-find-files where you can choose destination.

\subsection{Follow file after action}
\label{sec:follow-file-after}
A prefix arg on any of the action above, copy, rename, symlink, hardlink, will
allow you to follow the file.
For example, when you want to copy an elisp file somewhere and you want to compile it in this place,
hitting C-u RET will bring you in this place with the file already marked, you have just to go in action menu to
compile it.  

\subsection{In Buffer File Completion}
\label{sec:buff-file-compl}
In any buffer and even in minibuffer if you have enable recursive-minibuffer
\begin{verbatim}
(setq enable-recursive-minibuffers t)
\end{verbatim}
You can have completion with C-x C-f and then hit tab to choose action Complete at point \\ 
once you have found the filename you want.
\subsection{Create File}
\label{sec:create-file}
Navigate to the directory where you want to create your new file, then \\
continue typing the name of your file and type enter. \\
NOTE: If your filename ends with a / you will be prompt to create a new directory.

\subsection{Create Directory}
\label{sec:create-directory}
Navigate to the directory where you want to create your new directory, \\
then continue typing the name of new directory - Parents accepted - \\
and end it with / type enter, you will be prompt to create your new directory (possibly with parents).

\subsection{Ediff files}
\label{sec:ediff-files}
Well, that is easy to use, 
move cursor to a file, hit ediff in action menu, you will jump in
another anything-find-files to choose second file.
\subsection{Ediff merge files}
\label{sec:ediff-merge-files}
move cursor to a file, hit ediff merge in action menu, you will jump in
another anything-find-files to choose second file.

\subsection{Browse archive with avfs}
\label{sec:browse-archive-with}

If you have installed avfs (See: http://sourceforge.net/projects/avf) you can browse archives
in your directory .avfs once it is mounted with ``mountavfs''.

Just move on the archive filename and press C-z (persistent action) and you will see
in anything buffer the subdirectories of archive, just navigate inside as usual.

\subsection{Display with icons}
\label{sec:display-with-icons}

You can have a more fancy display showing icons for files and directories.
Just add in .emacs:
\begin{verbatim}
(setq anything-c-find-files-show-icons t)
\end{verbatim}
NOTE: This will slowdown anything-find-files unless you have a fast computer.

\section{Anything write buffer}
\label{sec:anyth-write-buff}
That is a replacement of standard write-buffer Emacs command with anything completion.
\section{Anything insert file}
\label{sec:anything-insert-file}
That is a replacement of standard insert-file Emacs command with anything completion.

\section{Anything M-x}
\label{sec:anything-m-x}
It is binded to f5 a M-x, you should bind it to M-x.

Features:\\
\begin{itemize}
\item You can use prefix arguments before or during M-x session
\end{itemize}
\begin{itemize}
\item C-z is a toggle documentation for this command
\end{itemize}
\begin{itemize}
\item The key binding of command are shown.
\end{itemize}

\section{Anything regexp}
\label{sec:anything-regexp}
This is a replacement of regexp-builder.
The groups are shown in a convenient way.

\subsection{Query replace regexp}
\label{sec:query-replace-regexp}
Write your regexp in anything-regexp, when it match what you want,
you can run query-replace from action menu.
NOTE:
Before running anything-regexp, you can select a region to work in, that will narrow this region
automatically. 
\subsection{Save regexp as sexp}
\label{sec:save-regexp-as}
When you use this, it will save your regexp for further use in lisp code,
with backslash duplicated.

\subsection{Save regexp as string}
\label{sec:save-regexp-as-1}
Save the regexp as you wrote it.

\section{Anything locate}
\label{sec:anything-locate}
First be sure you have a locate program installed on your system.
Most GNU/Linux distro come with locate included, you update or create the data base with
``updatedb'' command.

\subsection{Search files}
\label{sec:search-files}

To use, just launch 
\begin{verbatim}
M-x anything-locate (f5 a l)
\end{verbatim}

Then enter filename at prompt.
It will search this pattern entered also in directory and subdirectory names, to limit your search to basename,
add ``-b'' after pattern.
The search is performed on all files known in database, they maybe not exists anymore, so to limit to
really existing files add after pattern ``-e''.
To limit you search to specific number of results, use ``-n'' after your pattern with the number of results
you want.

Example:
\begin{verbatim}
Pattern: emacs -b -e -n 12
\end{verbatim}

\subsection{Launch grep}
\label{sec:launch-grep}

When search is done, you can search in a specific file or directory with grep that you will find in action menu (TAB).\\

\begin{itemize}
\item NOTE:You can now launch grep with (C-u) M-g s without switching to the action pannel.
\end{itemize}

\subsection{Windows specificity}
\label{sec:windows-specificity}

On Windows you should use Everything program that mimic locate, is very fast and don't need to
update database manually.
To use with anything-locate, you will need his command line named ``es''.
Be sure to modify the PATH environment variable, to include path to the directory that contain ``es''.
The arguments are the same than the ones in ``locate''.

\section{Anything Etags}
\label{sec:anything-etags}

\subsection{Create the tag file}
\label{sec:create-tag-file}


To use etags in Emacs you have first to create a TAGS file for your project with the etags shell command.
If your directory contains subdirectories use someting like:(e.g .el files)
\begin{verbatim}
find . -iregex .*\.el$ | xargs etags
\end{verbatim}
Otherwise
\begin{verbatim}
etags *.el
\end{verbatim}
is enough

For more infos see the man page of etags.

\subsection{Start anything etags}
\label{sec:start-anything-etags}


Now just using f5 a e will show you all entries.
If the project is big it take some time to load tag file but when it is done, next search will be very fast.
If you modify the TAGS file, use
\begin{verbatim}
C-u C-u f5 a e
\end{verbatim}
to refresh the tag cache.

To search the definition at point use just
\begin{verbatim}
C-u f5 a e
\end{verbatim}

\section{Firefox bookmarks}
\label{sec:firefox-bookmarks}

You will have to set firefox to import bookmarks in his html file book-marks.html. 
\begin{verbatim}
(only for firefox versions >=3)
\end{verbatim}
To achieve that, open ``about:config'' in firefox and double click on this line to enable value to true :
\begin{verbatim}
user_pref("browser.bookmarks.autoExportHTML", false);
\end{verbatim}
You should have now :
\begin{verbatim}
user_pref("browser.bookmarks.autoExportHTML", true);
\end{verbatim}
Now you can use
\begin{verbatim}
M-x anything-firefox-bookmarks
\end{verbatim}
To see your firefox bookmarks from Emacs.
When you are in firefox things are a little more complicated. You will
need wmctrl program and a script named ffbookmarks :
\begin{verbatim}
#!/bin/bash

wmctrl -xa emacs
emacsclient -e "(progn (anything-firefox-bookmarks) nil)" > /dev/null
wmctrl -xa firefox
exit 0
\end{verbatim}
Put this script somewhere in PATH and make it executable :
\begin{verbatim}
chmod +x ffbookmarks
\end{verbatim}

Firefox is not aware about this new protocol, you will have to instruct
it. See Firefox documentation or use firefox-protocol.el package you can get
here: \\
\url{http://mercurial.intuxication.org/hg/emacs-bookmark-extension/} \\
Install new protocol: \\
\begin{verbatim}
M-x firefox-protocol-installer-install
\end{verbatim}
to install new protocol ffbookmarks
Then install a bookmarklet in firefox : Right click on the bookmark
toolbar in firefox and add a new bookmark called ffbookmarks. Add this
instead of url: \\
\begin{verbatim}
javascript:location.href='ffbookmarks://localhost'
\end{verbatim}

Now when you click on ffbookmarks it will bring you in Emacs and allow
you to browse your bookmarks with anything.
\\
NOTE : emacs server need to be started in the running Emacs, see Emacs
documentation.

\section{Other tools}
\label{sec:other-tools}

In addition of what is described above, you will find a bunch of powerfull tools that come with anything-config.el.
Just browse the anything commands availables with anything-M-x.

Not complete.
\section{Usefuls extensions}
\label{sec:usefuls-extensions}
Not complete.

\section{Usefuls links}
\label{sec:usefuls-links}

You can have info about anything on Emacswiki (Sometime deprecated).\\
\url{http://www.emacswiki.org/emacs/Anything}\\

You can ask on the anything mailing-list by subscribing at:\\
\url{https://groups.google.com/group/emacs-anything?hl=en}

\end{document}

%%% Local Variables: 
%%% mode: latex
%%% TeX-master: t
%%% End: 

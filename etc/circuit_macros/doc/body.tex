\newcommand{\bq}{}
\newcommand{\pic}{{\bq pic}}
\newcommand{\dvips}{{\bq dvips}}
\newcommand{\Pic}{{\bq Pic}}
\newcommand{\gpic}{{\bq gpic}}
\newcommand{\Gpic}{{\bq Gpic}}
\newcommand{\Dpic}{{\bq Dpic}}
\newcommand{\tpic}{{\bq tpic}}
\newcommand{\mfpic}{{\bq mfpic}}
\newcommand{\groff}{{\bq groff}}
\newcommand{\xfig}{{\bq xfig}}
\newcommand{\Xfig}{{\bq Xfig}}
\newcommand{\PSTricks}{{\bq PSTricks}}
\newcommand{\MetaPost}{{\bq MetaPost}}
\newcommand{\Postscript}{{\bq Postscript}}
\newcommand{\dpic}{{\bq dpic}}
\newcommand{\Mfour}{{\bq m4}}
\newcommand{\TPGF}{{\bq Ti{\it k}z~PGF}}
\newcommand{\SVG}{{\bq SVG}}
\newcommand{\xection}[1]{\section[#1\ \dotfill]{#1}}
\newcommand{\NVL}{\\\hspace*{\parindent}}
%\newcommand{\lbrace}{{\tt\char123}}
%\newcommand{\rbrace}{{\tt\char125)}
%
  \tableofcontents
\enlargethispage{\baselineskip}
%
\xection{Introduction}
   \begin{quotation}\noindent
   Before every conference, I find Ph.D.s in on weekends running back
   and forth from their offices to the printer.  It appears that people
   who are unable to execute pretty pictures with pen and paper find it
   gratifying to try with a computer~\cite{Landauer95}.
   \end{quotation}

This document describes a set of macros, written in the \Mfour\ macro
language~\cite{KRm4}, for producing electric circuits and other
diagrams in \LaTeX\ documents.  The macros evaluate to drawing commands
in \pic, a line-drawing language~\cite{KRpic} that is simple to learn
and readily available.  The result is a system with the
advantages and disadvantages of \TeX\ itself, since it is macro-based
and non-wysiwyg, and since it uses ordinary character input.  The book
from which the above quotation is taken correctly points out that the
payoff can be in quality of diagrams at the price of the time spent in
learning how to draw them.

A collection of basic components and conventions for their internal
structure are described.  It is often convenient to customize elements
or to package combinations of them for particular drawings, so macros
such as these are only a starting point.

\xection{Using the macros}
This section describes the basic process of adding circuit diagrams to
\LaTeX\ documents to produce postscript or pdf files.  On some operating
systems, project management software with graphical interfaces can be used
to automate the process but the steps can be performed by a script,
makefile, or by hand for simple documents as described in
Section~\ref{Quickstart:}.

The diagram source file is preprocessed as illustrated in
Figure~\ref{Flowdiag}.  The predefined macros, followed by the diagram
source, are read by \Mfour.  The result is passed through a
\pic\ interpreter to produce {\tt .tex} output that can be inserted
into a {\tt .tex} document using the \verb|\input| command.

\begin{figure}[hbt]
   \input Flowdiag
   \caption{Inclusion of figures and macros in the \LaTeX\ document.
   Replacing \LaTeX\ with PDFlatex to produce pdf directly is also possible.}
   \label{Flowdiag}
   \end{figure}

\noindent
Depending on the \pic\ interpreter chosen, the choice of options,
and the choice of \LaTeX\ or PDFlatex, the
interpreter output may contain \tpic\ specials, \LaTeX\ graphics,
\PSTricks~\cite{pstricks} commands, \TPGF\ commands, or other formats.
These variations are described in Section~\ref{Alternative:}.

There are two principal choices of \pic\ interpreter.  One is~\dpic,
described later in this document.  An alternative \cite{gpic} is {\bq
gpic -t} together with a printer driver that understands \tpic\ 
specials, typically~\cite{dvips} {\bq dvips}.  In some installations,
\gpic\ is simply named \pic, but make sure that GNU \pic~\cite{gpic}
is being invoked rather than the older Unix \pic.
\Pic\ processors contain basic macro facilities, so some of the
concepts applied here require only a \pic\ processor.
%
By judicious use of macros, features of both \Mfour\ and \pic\ can be
exploited.
\iffalse
The fastidious reader might observe that there are three
languages being scrambled: \Mfour, \pic, and the \tpic, \TeX\ 
or other output, not to mention the meta-language of the macros, and
that this mixture might be a problem, but experience implies otherwise.
\fi

\subsection{Quick start}\label{Quickstart:}
The contents of file {\tt quick.m4} and resulting diagram are shown in
Figure~\ref{quick} to illustrate the language, to show several ways for
placing circuit elements, and to provide sufficient information for producing
basic labeled circuits.
\begin{figure}[hbt]
   \parbox{\textwidth}{\small\verbatiminput{quick.m4}}%
   \hfill\llap{\raise-1.15in\hbox{\input quick }}
   \vspace*{-\baselineskip}
   \caption{The file {\tt quick.m4} and resulting diagram.}
   \label{quick}
   \end{figure}

To process the file, make sure that the libraries {\tt libcct.m4} and
{\tt libgen.m4} are installed and readable.  Verify that \Mfour\ is installed.
Now there are at least two possibilities, described in the following
subsections, with somewhat simpler usage to be given in
Section~\ref{Simplified}.

\subsubsection{\protect{Processing with \dpic\ and \PSTricks\ or \TPGF}}
If you are using \dpic\  with \PSTricks,
type the following commands or put them into a script:
\begin{verbatim}
  m4 <path>pstricks.m4 <path>libcct.m4 quick.m4 > quick.pic
  dpic -p quick.pic > quick.tex
\end{verbatim}
where {\tt <path>} is the path to the {\tt libcct.m4} file.
Put \verb|\usepackage{pstricks}| in the main \LaTeX\ source file header and
the following in the body:
\begin{verbatim}
\begin{figure}[hbt]
   \centering
   \input quick
   \caption{Customized caption for the figure.}
   \label{Symbolic_label}
\end{figure}
\end{verbatim}
This distribution is compatible with the \TPGF\ drawing commands, which have
nearly the power of the \PSTricks\ package with the ability to produce 
{\tt pdf} output by running the
{\tt pdflatex} command instead of {\tt latex} on the input file.  The
commands are modified to read \verb|pgf.m4| and invoke the
\dpic~\verb|-g| option as follows:
\begin{verbatim}
  m4 <path>pgf.m4 <path>libcct.m4 quick.m4 > quick.pic
  dpic -g quick.pic > quick.tex
\end{verbatim}
The \LaTeX\ header should contain \verb|\usepackage{tikz}|, but the inclusion
statemensts are the same as for \PSTricks\ input.

In all cases the essential line is \verb|\input quick|, which
inserts the previously created file {\tt quick.tex}.
Then \LaTeX\ the document, convert to postscript typically using \dvips,
and print the result or view it using Ghostview.  The alternative
for \TPGF\ output of \verb|dpic -g| is to invoke PDFlatex.

\subsubsection{Processing with \gpic}
If your printer driver understands \tpic\ specials and
you are using \gpic\  (on some systems the \gpic\ command is {\tt pic}),
the commands are
\begin{verbatim}
  m4 <path>libcct.m4 quick.m4 > quick.pic
  gpic -t quick.pic > quick.tex
\end{verbatim}
The figure inclusion statements are as shown:
\begin{verbatim}
\begin{figure}[hbt]
   \input quick
   \centerline{\box\graph}
   \caption{Customized caption for the figure.}
   \label{Symbolic_label}
   \end{figure}
\end{verbatim}

\subsubsection{Simplifications}\label{Simplified}
\begin{itemize}
\item
These macros can be processed by \LaTeX-specific
project software and by graphic applications such as Cirkuit\cite{KDEApps2009}.
\item
If appropriate
{\tt include()} statements are placed at the top of the file {\tt quick.m4},
then the \Mfour\ commands illustrated above can be shortened to

\verb|m4 quick.m4 > quick.pic|

\noindent
For example, the following two lines can be inserted before or just after the
{\tt .PS} line:

{\tt include(<path>pstricks.m4)}\NVL
{\tt include(<path>libcct.m4)}

\noindent
where {\tt <path>} is the path to the folder containing the libraries.
Only the second line is necessary if \gpic\ is used or if the libraries
were installed so that \PSTricks\ is assumed by default.
Once one library file has been read, then the second and later include
lines can invoke the {\tt HOMELIB\_} macro for the path, and the above
two lines, for example, could be

{\tt include(<path>pstricks.m4)}\NVL
{\tt include(HOMELIB\_`'libcct.m4)}

\item
On some systems, setting the environment variable {\tt M4PATH} to
the library folder allows the above lines to be simplified to

{\tt include(pstricks.m4)}\NVL
{\tt include(libcct.m4)}

\item
In the absence of a need to examine the file {\tt quick.pic},
the commands for producing the {\tt .tex} file can be reduced
(provided the above inclusions have been made) to

\verb%m4 quick.m4 | dpic -p > quick.tex%

\item
When many files are to be processed, then a facility such as Unix ``make,''
which is also available in several PC versions, can be employed
to automate the commands given above.  On systems without such a
facility, a scripting language can be used.
\item  %Alternatively, you
You
can put
several diagrams into a single source file so that they can be
processed together.  Put each diagram in the body of a
\LaTeX\ macro, as shown:

\par
\verb|\newcommand{\diaA}{%|\NVL
\verb|.PS|\NVL
{\sl drawing commands}\NVL
\verb|.PE|\NVL
\verb|\box\graph }%  \box\graph not required for dpic|\NVL
\verb|\newcommand{\diaB}{%|\NVL
\verb|.PS|\NVL
{\sl drawing commands}\NVL
\verb|.PE|\NVL
\verb|\box\graph }%  \box\graph not required for dpic|\NVL
\noindent Process the file using \Mfour\ and \dpic\ or \gpic\ to produce
a {\tt.tex} file, insert this into the \LaTeX\ source using \verb|\input|, and 
invoke the macros at the appropriate places.

\item
It may be desirable to invoke \Mfour\ and \dpic\ automatically from the
document file, as shown:
\begin{verbatim}
\documentclass{article}
\usepackage{tikz}
\newcommand\mtopgf[1]{\immediate\write18{m4 <path>/pgf.m4 #1.m4 | dpic -g > #1.tex}}%
\begin{document} 
\mtopgf{PicA}
\input{PicA.tex} \par
\mtopgf{PicB}
\input{PicB.tex}
\end{document}
\end{verbatim}

The Unix-style path to \verb|pgf.m4| must be defined in this file,
sources \verb|PicA.m4| and \verb|PicB.m4| must contain any required
\verb|include| statements,
and the document file should be processed using
the command \verb|pdflatex -shell-escape| {\tt<}filename{\tt>}.
Alternatively, the first file read by \Mfour\ in the above definition
of \verb|\mtopgf| can be omitted but \verb|include(<path>pgf.m4)|
must be the first line of each source file.
This method processes each picture source every time \LaTeX\ is run, so for
large documents containing many graphics, the \verb|\mtopgf|
line could be commented out after debugging the corresponding graphic.
\end{itemize}

\subsection{The libraries}\label{Libraries}
One of the following configuration files must be read first by \Mfour,
depending on the output format (see Section~\ref{Alternative:}) and choice
of either \dpic\ or \gpic\ as postprocessor:
{\tt gpic.m4, mfpic.m4, mpost.m4, pgf.m4, postscript.m4, pstricks.m4, svg.m4,}
or {\tt xfig.m4}. The usual case for producing circuit diagrams is to read
{\tt pstricks.m4} or {\tt pgf.m4} first when \dpic\ is the postprocessor, and
{\tt gpic.m4} for GNU \pic.
Each of these files causes {\tt libgen.m4,} a collection of
multipurpose macros, to be read, and sets the macro
{\tt m4picprocessor} to the appropriate value.

The elementary circuit macros are contained in {\tt libcct.m4,} which
initially reads {\tt gpic.m4} unless the default has been otherwise set or
unless {\tt m4picprocessor} has been defined.  Consequently,
only {\tt libcct.m4} and the diagram source need be read by default.

Elementary logic circuits are defined in {\tt liblog.m4}, which causes
{\tt libcct.m4} to be read, so the latter need not be called explicitly
when logic gates are drawn.

Only the libraries necessary for a given diagram are required; thus,
{\tt libcct.m4} need not be read if the diagram does not contain a
circuit element.

The library {\tt lib3D.m4} manipulates scalar triples for colour
generation and three-dimensional projections.  The macros in file
{\tt darrow.m4} define one way of drawing double lines and arrows.

Some of the distributed example files contain other specialized macros,
such as for flow charts and other purposes.  Diagrams
in other domains than electric circuits could be drawn easily using
similar macros, but the temptation to write them has been resisted in
favour of attempting to treat one specialized area well.

\xection{\Pic\ essentials}

\Pic\ source is a sequence of lines in a file.
The first line of a diagram begins with {\tt .PS} with optional following
arguments, and the last line is normally {\tt .PE}.
Lines outside of these pass through the \pic\ processor unchanged.

The visible objects can be divided conveniently into two classes, the
{\em linear} objects {\tt line, arrow, spline, arc,} and the
{\em planar} objects {\tt box, circle, ellipse.}

The object {\tt move} is linear but draws nothing.  A compound object,
or {\tt block,} is planar and consists of a pair of square brackets enclosing
other objects, as described in Section~\ref{Compound}.
Objects can be placed using absolute coordinates or relative to other objects.

\Pic\ allows the definition of real-valued variables, which are alphameric
names beginning with lower-case letters, and computations using them.
Objects or locations on the diagram can be given symbolic names
beginning with an upper-case first letter.

\subsection{Manuals}
At the time of writing, the classic \pic\ manual~\cite{KRpic} can be obtained
from URL:

\verb|http://www.cs.bell-labs.com/10thEdMan/pic.pdf|

A more complete manual~\cite{Raymond95} is included in the GNU \groff\ 
package.  A compressed postscript version is available, at least
temporarily, with these circuit files.

In both of the above manuals, explicit use of {\tt *roff} string and font
constructs should be replaced by their \LaTeX\ equivalents as necessary.
Further explanation is available, for example, from the \gpic\ 
``man'' page, part of the GNU \groff\  package.

Examples of use of the circuit macros in an electronics
course are available on the web~\cite{Clark99}.

For a discussion of ``little languages'' for document production, and
of \pic\ in particular, see Chapter~9 of \cite{Bentley88}.  Chapter~1
of \cite{Goossens97} also contains a brief discussion of this and
other languages.

\subsection{The linear objects: {\tt line, arrow, spline, arc}}
A line can be drawn as follows:

{\tt line from} {\sl position} {\tt to} {\sl position}

\noindent
where {\sl position} is defined below or

{\tt line} {\sl direction} {\sl distance}

\noindent
where {\sl direction} is one of {\tt up,} {\tt down,} {\tt left,}
{\tt right.}  When used with the \Mfour\ macros described here, it is
preferable to add an underscore: {\tt up\_,} {\tt down\_,} {\tt left\_,}
{\tt right\_.}  The {\sl distance} is a number or expression
and the units are inches, but the assignment

{\tt scale = 25.4}

\noindent
has the effect of changing the units to millimetres,
as described in Section~\ref{scaling:}.

Lines can also be drawn to any distance in any direction.  The example,

{\tt line up\_ 3/sqrt(2) right\_ 3/sqrt(2) dashed}

\noindent
draws a line 3 units long from the current location,
at a $45^\circ$ angle above horizontal.
Lines (and other objects) can be specified as {\tt dotted,} {\tt dashed,} or
{\tt invisible,} as above.

The construction

{\tt line from A to B chop x}

\noindent
truncates the line at each end by {\tt x} (which may be negative)
or, if {\tt x} is omitted, by
the current circle radius, which is convenent when A and B are
circular graph nodes, for example.  Otherwise

{\tt line from A to B chop x chop y}

\noindent
truncates the line by {\tt x} at the start and {\tt y} at the end.

Any of the above means of specifying line (or arrow) direction and length
will be called a {\em linespec.}

Lines can be concatenated.  For example, to draw a triangle:

{\tt line up\_ sqrt(3) right\_ 1 then down\_ sqrt(3) right\_ 1 then left\_ 2}

\subsection{Positions}
A {\sl position} can be defined by a coordinate pair, e.g.\ {\tt 3,2.5},
more generally using parentheses by {\tt (}{\sl expression, expression}{\tt )},
as a sum or difference as
{\tt{\sl position} $+$ ({\sl expression, expression})},
or by the construction {\tt (}{\sl position, position}{\tt)},
the latter taking the $x$-coordinate from the first
position and the $y$-coordinate from the second.  A position can be
given a symbolic name beginning with an upper-case letter,
e.g.\ {\tt Top:~(0.5,4.5)}.  Such a definition does not affect the calculated
figure boundaries.  The current position {\tt Here} is always defined and
is equal to $(0,0)$ at the beginning of a diagram or block.
The coordinates of a position are accessible, e.g.\ {\tt Top.x} and
{\tt Top.y} can be used in expressions.  The center, start, and end of
linear objects (and the defined points of other objects as described below)
are predefined positions, as shown in the following example,
which also illustrates how to refer to a previously drawn element if it has
not been given a name:

{\tt line from last line.start to 2nd last arrow.end then to 3rd line.center}

Objects can be named (using a name commencing with an upper-case letter),
for example:

{\tt Bus23: line up right}

\noindent
after which, positions associated with the object can be referenced using the
name; for example:

{\tt arc cw from Bus23.start to Bus23.end with .center at Bus23.center}

An arc is drawn by specifying its rotation, starting point, end point, and
center, but sensible defaults are assumed if any of these are omitted.
Note that

{\tt arc cw from Bus23.start to Bus23.end}

\noindent
does {\em not} define the arc uniquely; there are two arcs that satisfy this
specification.
This distribution includes the \Mfour\ macros

{\tt arcr( {\sl position, radius, start radians, end radians})
\hfill\break\indent
     arcd( {\sl position, radius, start degrees, end degrees})
\hfill\break\indent
     arca( {\sl chord linespec,} ccw|cw, {\sl radius, modifiers})}

\noindent to draw uniquely defined arcs.  For example,

{\tt arcd((1,1),2,0,-90) -> cw}

\noindent draws a clockwise arc with centre at $(1,1),$ radius $2,$
 from $(3,1)$ to $(1,-1),$ and

{\tt arca(from (1,1) to (2,2),,1,->)}

\noindent draws an acute-angled arc with arrowhead on the chord defined by the
first argument.

The linear objects can be given arrowheads at the start, end, or both ends,
for example:

{\tt line dashed <- right 0.5\hfill\break
\hspace*{\parindent}%
arc <-> height 0.06 width 0.03 ccw from Here to Here+(0.5,0)
 \char92\hfill\break
\hspace*{2\parindent}%
   with .center at Here+(0.25,0)\hfill\break
\hspace*{\parindent}%
spline -> right 0.5 then down 0.2 left 0.3 then right 0.4}

The arrowheads on the arc above have had their shape adjusted using the
{\tt height} and {\tt width} parameters.

\subsection{The planar objects: {\tt box, circle, ellipse}, and text}
Planar objects are drawn by specifying the width, height, and position, thus:

{\tt A: box ht 0.6 wid 0.8 at (1,1)}

\noindent
after which, in this example, the position {\tt A.center} is defined,
and can be referenced simply as {\tt A}.  In addition, the
compass corners {\tt A.n,} {\tt A.s,} {\tt A.e,} {\tt A.w,} {\tt A.ne,}
{\tt A.se,} {\tt A.sw,} {\tt A.nw} are automatically defined, as are
the dimensions {\tt A.height} and {\tt A.width.}
Planar objects can also be placed by specifying the location of a defined
point; for example, two touching circles can be drawn as shown:

{\tt circle radius 0.2\hfill\break 
\hspace*{\parindent}%
circle diameter (last circle.width * 1.2) with .sw at last circle.ne}

The planar objects can be filled with gray or colour. For example, the line

{\tt box dashed fill}

\noindent produces a dashed box filled with a medium gray by default.
The gray density can be controlled using the \verb|fill_(|{\sl number}\verb|)|
macro, where $0\leq \hbox{\sl number}\leq 1$, with $0$ corresponding to
black and $1$ to white. 

Basic colours for lines and fills are provided by \gpic\  and \dpic,
but more elaborate line and fill styles can be incorporated, depending
on the printing device, by inserting {\tt \char92 special} commands or
other lines beginning with a backslash in the drawing code.  In fact,
arbitrary lines can be inserted into the output using

{\tt command "}{\sl string}{\tt "}

\noindent where {\sl string} is one or more lines to be inserted.

Arbitrary text strings, typically meant to be typeset by \LaTeX, are
delimited by double-quote characters and occur in two ways.  The first
way is illustrated by

\verb|"\large Resonances of $C_{20}H_{42}$"|
 \verb|wid |{\sl x}\verb| ht |{\sl y}\verb| at |{\sl position}

\noindent
which writes the typeset result, like a box, at {\sl position} and tells
\pic\ its size.  The default size assumed by \pic\ is given by parameters
{\tt textwid} and {\tt textht} if it is not specified as above.
The exact typeset size of formatted text can be obtained
as described in Section~\ref{Interaction:}.  The second occurrence
associates one or more strings with an object, e.g., the following writes
two words, one above the other, at the centre of an ellipse:

\verb|ellipse "\bf Stop" "\bf here"|

\noindent
The C-like \pic\ function
 {\tt sprintf("{\sl format string}",{\sl numerical arguments})}
is equivalent to a string.

\subsection{Compound objects}\label{Compound}
A compound object is a group of statements enclosed in square
brackets.  Such an object is placed by default as if it were a box, but
it can also be placed by specifying the final position of a defined point.
A defined point is the center or compass courner of the bounding box
of the compound object or one of its internal objects.
Consider the last line of the code fragment shown:

\noindent%
\verb|  Ands: [ right_|\\
\verb|          And1: AND_gate|\\
\verb|          And2: AND_gate at And1 - (0,And1.ht*3/2)|\\
\verb|          |$\ldots$\\
\verb|        ] with .And2.In1 at| {\sl position} % (K.x,IC5.Pin9.y)|

The two gate macros evaluate to compound objects containing {\tt Out},
{\tt In1}, and other locations.  The final positions of all objects
between the square brackets are determined in the last line by
specifying the position of {\tt In1} of gate {\tt And2}.

\subsection{Other language facilities}

All objects have default sizes, directions, and other characteristics,
so part of the specification of an object can sometimes be profitably
omitted.

Another possibility for defining positions is 

{\sl expression} {\tt of the way between} {\sl position}
 {\tt and} {\sl position}

\noindent%
which is abbreviated as

{\sl expression} {\tt <} {\sl position} {\tt ,} {\sl position} {\tt >}

\noindent%
but care has to be used in processing the latter construction with \Mfour,
since the comma may have to be put within quotes, {\tt `,'} to distinguish it
from the {\tt m4} argument separator.

Positions can be calculated using expressions containing variables.
The scope of a position is the current block.  Thus, for example,

{\tt
  theta = atan2(B.y-A.y,B.x-A.x)

  line to Here+(3*cos(theta),3*sin(theta)).
  }

Expressions are the usual algebraic combinations of primary quantities:
constants, environmental parameters such as {\tt scale,} variables,
horizontal or vertical coordinates of terms such as
{\sl position}{\tt.x} or {\sl position}{\tt.y},
dimensions of \pic\ objects, e.g.\ {\tt last circle.rad}.
The elementary algebraic operators are
{\tt +, -, *, /, \%, =, +=, -=, *=, /=,} and {\tt \%=,}
similar to the C language.

The logical operators {\tt ==, !=, <=, >=, >,} and {\tt <} apply to
expressions, and strings can be tested for equality or inequality.  A
modest selection of numerical functions is also provided: the
single-argument functions {\tt sin, cos, log, exp, sqrt, int}, where
{\tt log} and {\tt exp} are base-10, the two-argument functions
{\tt atan2, max, min,} and the random-number generator {\tt rand()}.
Other functions are also provided using macros.

A \pic\ manual should be consulted for details, more examples, and
other facilities, such as the branching facility

\verb|if |{\sl expression}\verb| then { |{\sl anything} 
  \verb|} else { |{\sl anything}\verb| }|,

\noindent%
the looping facility

\verb|for |{\sl variable}\verb| = |{\sl expression}\verb| to |% 
{\sl expression}\verb| by |{\sl expression}\verb| do { |%
{\sl anything}\verb| }|,

\noindent%
operating-system commands, \pic\ macros, and external file inclusion.

\xection{Two-terminal circuit elements}\label{Basictwo}
There is a fundamental difference between two-terminal elements, which
are drawn as directed linear objects, and other elements, which are
compound objects as described in Section~\ref{Compound}.  The
two-terminal element macros follow a set of conventions described in
this section, and other elements will be described in Section
\ref{Other}.

\subsection{Circuit and element basics}
First, the arguments of all drawing macros have default values, so that
only arguments that differ from these values need be specified.  The
arguments are given in Section~\ref{defines}.

Consider the resistor shown in Figure~\ref{BigResistor},
which also serves as an example of \pic\ commands.
The first part of the source file for this figure is as follows:

{\small \input BigResistor1.verb }

\begin{figure}[hbt]
   \input BigResistor
   \caption{Resistor named {\tt R1}, showing the size parameters,
     enclosing block, and predefined positions.}
   \label{BigResistor}
   \end{figure}
The lines of Figure~\ref{BigResistor}
and the remaining source lines of the file are explained below:
\begin{itemize}
\item The first line invokes an almost-empty macro {\tt cct\_init} that
   initializes local variables needed by some circuit-element macros.  This
   macro can be customized to set line thicknesses, maximum page sizes, scale
   parameters, or other global quantities as desired.
\item The body dimensions of two-terminal elements are multiples of the macro
   {\tt dimen\_}, which evaluates by default to {\tt linewid}, a
   \pic\ environmental parameter with default value 0.5\,in.  The default
   length of an element is {\tt elen\_}, which is {\tt dimen\_*3/2}.
   Setting {\tt linewid} to 2.0 as in the example means that the default element
   length becomes 3.0\,in.
   For resistors, the length of the body is {\tt dimen\_/2,} and the
   width is {\tt dimen\_/6.} All of these values can be customized.
   Element scaling is discussed further in Section~\ref{scaling:}.
\item The macro {\tt linethick\_} sets the default thickness of subsequent
   lines (to 2.0\,pt in the example).
   In the \Mfour\ language, macro arguments are written within parentheses
   following the macro name, with no space between the name and the
   opening parenthesis.  Lines can be broken before a macro argument
   because \Mfour\ ignores white space immediately preceding arguments.
\item The two-terminal element macros expand to sequences of drawing commands
   that begin with {\tt `line invis {\sl linespec}'},
   where {\tt\sl linespec} is the first argument of the macro if it
   is non-blank, otherwise the line is drawn a distance
   {\tt elen\_} in the current direction, which is to the right by
   default. All this is handled by the macro {\tt eleminit\_},
   which also calculates the length and angle of the invisible line for
   later use.  The invisible line is first drawn, then the element is drawn
   on top of it.
   The element---rather the initial invisible line---can
   be given a name, {\tt R1} in the example, so that positions
   {\tt R1.start}, {\tt R1.centre}, and {\tt R1.end} are defined as shown.
\item The element body is enclosed by a block, which can be
   used to place labels around the element.  The block
   corresponds to an invisible rectangle with horizontal top and bottom lines,
   regardless of the direction in which the element is drawn.  A
   dotted box has been drawn in the diagram to show the block boundaries.
\item The last sub-element, identical to the first in two-terminal
   elements, is an invisible line that can be referenced later to
   place labels or other elements.  This might be over-kill.  If you create
   your own macros you might choose simplicity over generality, and only
   include visible lines.
  \end{itemize}

To produce Figure~\ref{BigResistor}, the following embellishments
were added after the previously shown source:
{\small \input BigResistor2.verb }

\begin{itemize}
\item The line thickness is set to the default thin value of \hbox{0.4\,pt},
   and the box displaying the element body block is drawn.  Notice how the
   width and height can be specified, and the box centre positioned at
   the centre of the block.
\item The next paragraph draws two objects, a spline with an arrowhead,
   and a string left justified at the end of the spline.  Other
   string-positioning modifiers than {\tt ljust} are {\tt rjust,}
   {\tt above,} and {\tt below.} Lines to be read by \pic\ can be
   continued by putting a backslash as the rightmost character. 

\item The last paragraph invokes a macro for dimensioning diagrams.
   \end{itemize}

\subsection{The two-terminal elements}
Figures~\ref{CctTable}--\ref{Switches} are tables of the two-terminal
elements.  Several elements are included more than once to illustrate
some of their arguments, which are listed in Section~\ref{defines}.
\begin{figure}[b!]
   \input CctTable
   \caption{Two-terminal elements, showing some variations.}
   \label{CctTable}
    \end{figure}
\begin{figure}[t!]
   \input Sources
   \caption{Sources and source-like elements.}
   \label{Sources}
   \end{figure}
\begin{figure}[t!]
   \input Diodes
   \caption{Variants of
     {\tt diode({\sl linespec},B|D|K|L|LE[R]|P[R]|S|T|V|Z,[R][E])}.}
   \label{Diodes}
   \end{figure}
\begin{figure}[t!]
   \input Fuses
   \caption{The
     {\tt fuse({\sl linespec}, A|dA|B|C|D|E|S|HB|HC, {\sl wid}, {\sl ht})}
     and {\tt cbreaker({\sl linespec},L|R,D)} macros.}
   \label{Fuses}
   \end{figure}
\begin{figure}[t!]
   \input AmpTable
   \caption{Amplifier, delay, and integrator.}
   \label{AmpTable}
   \end{figure}
\begin{figure}[t!]
   \input Switches
   \caption{The basic
     {\tt switch({\sl linespec},L|R,[O|C][D],B)}
     macro and more elabourate
     {\tt dswitch({\sl linespec},R,W[ud]B[K]{\sl chars})} macro, with
     drawing direction {\tt right\_}.
     Setting the second argument to {\tt R} produces a mirror
     image with respect to the drawing direction.
     The macro {\tt switch(,,,D)} is a wrapper for the comprehensive
     {\tt dswitch} macro.}
   \label{Switches}
   \end{figure}

The first argument of the two-terminal elements, if included, defines
the invisible line along which the element is drawn.  The other
arguments produce variants of the default elements.  Thus, for example,

{\tt resistor(up\_ 1.25,7)}

\noindent%
draws a resistor 1.25 units long up from the current position, with $7$
vertices per side.
The macro {\tt up\_} evaluates to {\tt up} but also resets the current
directional parameters to point up.
%\pagebreak

Most of the two-terminal elements are oriented; that is, they have
a defined polarity.  Several element macros include an argument
that reverses polarity, but there is also a more general mechanism.
The first argument of the macro

{\tt reversed(`}{\sl macro name}{\tt',}{\sl macro arguments}{\tt )}

\noindent
is the name of a two-terminal element in quotes, followed by the
element arguments.  The element is drawn with reversed direction.
Thus,

{\tt diode(right\_ 0.4); reversed(`diode',right\_ 0.4)}

\noindent
draws two diodes to the right, but the second one points left.
Similarly, the macro

{\tt resized(}{\sl factor},`{\sl macro name}',{\sl macro arguments}{\tt )}

\noindent
can be used to resize the body of an element by temporarily multiplying
the {\tt dimen\_} macro by {\sl factor}. More general resizing should be
done by redefining {\tt dimen\_} as described in Section~\ref{Circuitscaling:}.
These two macros can be nested; the following scales the above example
by 1.8, for example

{\tt resized(1.8,`diode',right\_ 0.4);}
{\tt resized(1.8,`reversed',`diode',right\_ 0.4)}

%\pagebreak
Figure~\ref{Variable} shows some two-terminal elements with
arrows or lines overlaid to indicate variability using the macro
{\tt variable(`}{\sl element}{\tt',{\sl type},{\sl angle},{\sl length})},
where {\sl type} is one of {\tt A, P, L, N,} with {\tt C} or {\tt S}
optionally appended to indicate continuous or stepwise variation.
Alternatively, this macro
can be invoked similarly to the label macros in
Section~\ref{Labels:} by specifying an empty first argument;
thus, the following line draws the resistor in Figure~\ref{Variable}:

   {\tt resistor(down\_ dimen\_); variable(,uN)}

\begin{figure}[h!t]
\vspace*{-\baselineskip}
   \input Variable
   \caption{Illustrating
   {\tt variable(`{\sl element}',[A|P|L|[u]N][C|S],{\sl angle},{\sl length})}.
   For example,\break {\tt variable(`capacitor(down\_ dimen\_)')} draws
   the leftmost capacitor shown above, and {\tt variable(`resistor(down\_
   dimen\_)',uN)} draws the resistor.  The default angle is
   45${}^{\circ}$, regardless of the direction of the element.  The array
   on the right shows the effect of the second argument.}
   \label{Variable}
   \end{figure}

Figure~\ref{Emarrows} contains arrows for indicating
radiation effects.
\begin{figure}[h!t]
   \input Emarrows
   \caption{Radiation arrows: {\tt em\_arrows({\sl type, angle, length})}}
   \label{Emarrows}
   \end{figure}
The arrow stems are named {\sl A1}, {\sl A2},
and each pair is drawn in a \verb|[]| block, with
the names {\sl Head} and {\sl Tail} defined to
aid placement near another device.  The second argument specifies
absolute angle in degrees (default 135 degrees).
\enlargethispage{\baselineskip}

\subsection{Branch-current arrows}
Arrowheads and labels can be added to conductors using basic
\pic\ statements.  For example, the following line adds a labeled
arrowhead at a distance {\tt alpha} along a horizontal line that has
just been drawn.  Many variations of this are possible:

  \verb|arrow right arrowht from last line.start+(alpha,0) "$i_1$" above|

Macros have been defined to simplify the labelling of two-terminal
elements, as shown in Figure~\ref{currents}.
The macro

   {\tt b\_current({\sl label,} above\_|below\_, In|O[ut], Start|E[nd],
   {\sl frac})}

\noindent
draws an arrow from the start of the last-drawn two-terminal element
{\sl frac} of the way toward the body.
\begin{figure}[h!t]
   \input currents
   \caption{Illustrating {\tt b\_current, larrow,} and {\tt rarrow}.
      The drawing direction is to the right.}
   \label{currents}
   \end{figure}

If the fourth argument is {\tt End},the arrow is drawn from the end
toward the body.
If the third element is {\tt Out}, the arrow is drawn outward from the body.
The first argument is the desired label, of which the default position is
the macro {\tt above\_,} which evaluates to {\tt above} if the current
direction is right or to {\tt ljust, below, rjust} if the current
direction is respectively down, left, up.  The label is assumed to be
in math mode unless it begins with {\tt sprintf} or a double quote, in which
case it is copied literally.  A non-blank second argument specifies the
relative position of the label with respect to the arrow, for example
{\tt below\_,} which places the label below with respect to the current
direction.  Absolute positions, for example {\tt below} or {\tt ljust},
also can be specified.

For those who prefer a separate arrow to indicate the reference
direction for current, the macros {\tt larrow({\sl label}, ->|<-,{\sl dist})}
and {\tt rarrow({\sl label}, ->|<-,{\sl dist})} are provided.  The label is
placed outside the arrow as shown in Figure~\ref{currents}.  The first
argument is assumed to be in math mode unless
it begins with {\tt sprintf} or a double
quote, in which case the argument is copied literally.  The third argument
specifies the separation from the element.
%\begin{figure}[hbt]
%   \input lrarrows
%   \caption{The {\tt larrow} and {\tt rarrow} macros draw
%    reference-direction arrows adjacent to the element.}
%   \label{lrarrows}
%   \end{figure}

\subsection{Labels}\label{Labels:}
   Macros for labeling two-terminal elements are included:
\par
{\tt
   llabel(} {\sl arg1,arg2,arg3} {\tt )
      \hfill\break\hspace*{\parindent}%
   clabel(} {\sl arg1,arg2,arg3} {\tt )
      \hfill\break\hspace*{\parindent}%
   rlabel(} {\sl arg1,arg2,arg3} {\tt )
      \hfill\break\hspace*{\parindent}%
   dlabel(} {\sl long,lat,arg1,arg2,arg3} {\tt )
   }

The first macro places the three arguments, which are treated as math-mode
strings, on the left side of the element block {\em with respect to the
current direction:} {\tt up, down, left, right.}
The second places the arguments along the centre, and the third along the
right side.
Thus a simple circuit example with labels is shown in Figure~\ref{Loop}.
The macro {\tt dlabel} performs these functions for an
obliquely drawn element, placing the three macro arguments at
{\tt vec\_(-long,lat),} {\tt vec\_(0,lat),} and {\tt vec\_(long,lat)}
respectively relative to the centre of the element.  Labels beginning
with {\tt sprintf} or a double quote are copied literally rather than
assumed to be in math mode.
\begin{figure}[ht]
   \vspace*{-\baselineskip}
   \parbox{4in}{\small \verbatiminput{Loop.m4}}%
   \hfill\raise-0.5in\hbox{\input Loop }
   \vspace*{-\baselineskip}
   \caption{A loop containing labeled elements, with its source code.}
   \label{Loop}
   \end{figure}

\xection{Other circuit elements}\label{Other}
Many basic elements are not two-terminal.  These elements are usually enclosed
in a \verb|[ ]| block, and contain named interior locations and components.
In some cases, an
invisible line determining length and direction (but not position) can
be specified by the first argument, as for the two-terminal elements.
Instead of positioning by the first line, the enclosing block must be
placed by using its compass corners, thus:
  {\sl element} {\tt with} {\sl corner} {\tt at} {\sl position} 
or, when the block contains a predefined location, thus:
  {\sl element} {\tt with} {\sl location} {\tt at} {\sl position}.
A few macros are positioned with the first argument;
the {\tt ground} macro, for example:
  {\tt ground(}{\tt at} {\sl position}{\tt ).} 

Nearly all elements drawn within blocks can be customized by adding an
extra argument, which is executed as the last item within the block.

The macro {\tt
   potentiometer({\sl linespec},{\sl cycles},{\sl fractional pos},{\sl length},$
    \ldots$)},
shown in Figure~\ref{Potentiometers},
first draws a resistor along the specified line, then adds arrows for taps
at fractional positions along the body, with default or specified length.
A negative length draws the arrow from the right of the current drawing
direction.
\begin{figure}[ht!]
   \input Potentiometers
   \caption{Default and multiple-tap potentiometer.}
   \label{Potentiometers}
   \end{figure}

The ground symbol is shown in Figure~\ref{Grounds}.
The first argument specifies position; for example, the two lines shown
have identical effect:

{\tt move to (1.5,2); ground

ground(at (1.5,2)) }

\noindent The second argument truncates
the stem, and the third defines the symbol type.
The fourth argument specifies the angle at which the symbol is drawn,
with D (down) the default.
\begin{figure}[ht!]
   \input Grounds
   \caption{The 
     {\tt ground( at }{\sl position}{\tt, T, N|F|S|L|P|E, U|D|L|R|angle )}
     macro.}
   \label{Grounds}
   \end{figure}

The arguments of the macro
{\tt antenna( at }{\sl position}{\tt, T, A|L|T|S|D|P|F, U|D|L|R|angle )}\break
shown in Figure~\ref{Antennas} are similar to those of {\tt ground}.
\enlargethispage{\baselineskip}
\begin{figure}[h!t]
   \input Antennas
   \caption{Antenna symbols, with macro arguments shown above and
     terminal names below.}
   \label{Antennas}
   \end{figure}

Figure~\ref{Opamp} illustrates the macro
{\tt opamp({\sl linespec, - label, + label, size,} {\sl chars})}\label{OPAMP}.
The element is enclosed in a block
containing the predefined internal locations shown.
These locations can be referenced in later
commands, for example as `{\tt last [].Out}.'
The first argument defines the direction and length of the opamp, but the
position is determined either by the enclosing block of the opamp,
or by a construction such as `{\tt opamp with .In1 at Here}', which places
the internal position {\sl In1} at the specified location.
There are optional second and third arguments for which the defaults
are {\tt \char92{}scriptsize\$-\$} and {\tt \char92{}scriptsize\$+\$}
respectively, and the fourth argument changes the size of the opamp.
The fifth argument is a string of characters.  {\tt P}
adds a power connection, {\tt R} exchanges the second and
third entries, and {\tt T} truncates the opamp point.

\begin{figure}[h!t]
   \input Opamp
   \caption{Operational amplifiers.  The {\tt P} option adds
     power connections.  The second and third arguments can be used
     to place and rotate arbitrary text at {\tt In1} and {\tt In2}.}
   \label{Opamp}
   \end{figure}

Typeset text associated with circuit elements is not rotated by default,
as illustrated by the second and third opamps in Figure~\ref{Opamp}.
The {\tt opamp} labels can be rotated if necessary by 
using postprocessor commands (for example \PSTricks\ \verb|\rput|)
as second and third arguments.

The code in Figure~\ref{oax} places an opamp with three connections.
\begin{figure}[h!t]
   \parbox{4in}{\small \verbatiminput{oaxbody.m4}}%
   \quad\raise-0.2in\hbox{\input oax }%
   \vspace{-\baselineskip}
   \caption{A code fragment invoking the
    {\tt opamp({\sl linespec},-,+,{\sl size},[R][P])} macro.}
   \label{oax}
   \end{figure}

Figure~\ref{Xform} shows variants of the transformer macro,
which has predefined internal locations
{\sl P1,} {\sl P2,} {\sl S1,} {\sl S2,} {\sl TP,} and {\sl TS.}
The first argument
specifies the direction and distance from {\sl P1} to {\sl P2}, with
position determined by the enclosing block as for opamps.  The second
argument places the secondary side of the transformer to the left
or right of the drawing direction.  The optional third and fifth arguments
specifies the number of primary and secondary arcs respectively.
If the fourth argument string contains an {\tt A}, the iron core
is omitted, and if it contains a {\tt W}, wide windings are drawn.
\begin{figure}[h!t]
   \input Xform
  \caption{The {\tt transformer({\sl linespec},L|R,{\sl np},[A][W|L],{\sl ns})}
     macro (drawing direction {\tt down}), showing predefined terminal
     and centre-tap points.}
   \label{Xform}
   \end{figure}
\iffalse
A transformer with four connections is illustrated
in Figure~\ref{tran}.
\begin{figure}[hbt]
   \parbox{4in}{\small \verbatiminput{tranbody.m4}}%
   \quad\raise-0.2in\hbox{\input tran }%
   \vspace{-\baselineskip}
   \caption{Showing the
    {\tt transformer({\sl linespec},L|R,np,A,ns)} macro.}
   \label{tran}
   \end{figure}
\fi

Figure~\ref{Audio} shows some audio devices, defined in {\tt []} blocks,
with predefined internal locations as shown.
The first argument specifies the device orientation.
Thus,

{\tt S: speaker(U) with .In2 at Here}

\noindent
places an upward-facing speaker with input {\sl In2} at the
current location.
\begin{figure}[ht!]
   \input Audio
   \caption{Audio components:
   {\tt speaker(U|D|L|R|{\sl degrees},{\sl size},{\sl type}),
     bell, microphone, buzzer,
     earphone}, with their internally named positions and components.}
   \label{Audio}
   \end{figure}

The {\tt nport({\sl box specs {\tt[;} other commands{\tt]},
  nw, nn, ne, ns, space ratio, pin lgth, style})}
macro is shown in Figure~\ref{Nport}.
The macro begins with the line \verb|define(`nport',`[Box: box `$1'|,
so the first argument is a box specification, such as size or fill parameters,
or text, possibly followed by other embellishments for the box, such as labels.
The second to fifth arguments specify the number of ports
(pin pairs) to be drawn respectively on the west, north, east, and south
sides of the box.  The end of each pin has a name corresponding to the
side, port number and $a$ or $b$ pin, as shown.
The sixth argument
specifies the ratio of port width to inter-port space, the seventh is
the pin length, and setting the eighth argument to {\tt N} omits the pin
dots.
The macro ends with \verb|`$9']')|, so that a ninth argument can be used
to add further customizations within the enclosing block.
\begin{figure}[h!b]
   \input Nport
   \caption{The {\tt nport} macro draws a sequence of pairs of named pins
     on each side of a box.  The pin names are shown.  The default is a twoport.
     The {\tt nterm} macro draws single pins instead of pin pairs.}
   \label{Nport}
%\vspace*{-\baselineskip}
   \end{figure}

The {\tt nterm({\sl box specs, nw, nn, ne, ns, pin lgth, style})} macro
illustrated in Figure~\ref{Nport} is similar to the {\tt nport} macro but
has one fewer argument, draws single pins instead of pin pairs, and
defaults to a 3-terminal box.

Many custom labels or added elements may be required, particularly for
2-ports\label{NPORTS}. These elements can be added using the first
argument and the ninth of the {\tt nport} macro.
For example, the following code adds a pair of labels to the box
immediately after drawing it but within the enclosing block:

{\tt nport(; {`"${}0$"' at Box.w ljust; `"$\infty$"' at Box.e rjust})}

If this trick were to be used extensively, then the following custom wrapper
would save typing, add the labels, and pass all arguments to
{\tt nport}:

\begin{verbatim}
define(`nullor',`nport(`$1'
  {`"${}0$"' at Box.w ljust
   `"$\infty$"' at Box.e rjust},shift($@))')
\end{verbatim}

The above example and the related gyrator macro are illustrated in
Figure~\ref{NLG}. 
\begin{figure}[h!t]
   \input NLG
   \caption{The {\tt nullor} example and the {\tt gyrator}
    macro are customizations of the {\tt nport} macro.}
   \label{NLG}
   \end{figure}

\pagebreak
A basic winding macro for magnetic-circuit sketches and similar figures
is shown in Figure~\ref{Windings}.
For simplicity, the complete spline
is first drawn and then blanked in appropriate places using the background
(core) color (\verb!lightgray! for example, default \verb!white!).
\begin{figure}[h!]
 \vspace{-\baselineskip}
   \input Windings
   \vspace{-1ex}
   \caption{The {\tt winding(L|R, diam, pitch, turns, core wid, core color)}
     macro draws a coil with axis along the current drawing direction.
     Terminals {\tt T1} and {\tt T2} are defined.
     Setting the first argument to {\tt R} draws a right-hand winding.}
   \label{Windings}
   \end{figure}
\enlargethispage{\baselineskip}

Figure~\ref{Relay} shows the macro {\tt contact(O|C, R)}
which contains predefined locations {\sl P, C, O}
for the armature and normally closed and normally
open terminals.  The macro {\tt relay(}{\sl poles}{\tt, O|C, R)}
defines coil terminals {\sl V1, V2} and contact
terminals {\sl P$_i$, C$_i$, O$_i$.} 
\begin{figure}[ht]
   \input Relay
   \vspace{-1ex}
   \caption{The {\tt contact(O|C,R)} and {\tt relay(}{\sl poles}{\tt,O|C,R)}
     macros (default direction right).}
   \label{Relay}
%\vspace{-\baselineskip}
   \end{figure}

Figure~\ref{Bip} shows the variants of bipolar transistor macro
{\tt bi\_tr({\sl linespec},L|R,P,E)}
which contains predefined internal locations {\sl E},
{\sl B}, {\sl C}.
\begin{figure}[b!]
   \input Bip
   \caption{Bipolar transistor variants (current direction upward).}
   \label{Bip}
   \end{figure}
The first argument defines the distance and direction
from {\sl E} to {\sl C,} with location determined by the enclosing
block as for other elements, and the base placed to the left or right of the
current drawing direction according to the second argument.  Setting the third
argument to `{\tt P}' creates a PNP device instead of NPN, and setting the
fourth to `{\tt E}' draws an envelope around the device.

The code fragment example in Figure~\ref{bitr} places a bipolar transistor,
connects a ground to the emitter, and connects a resistor to the collector.
\begin{figure}[h!]
\vspace*{-\baselineskip}
   \quad\quad\parbox{4in}{\small \verbatiminput{bitrbody.m4}}%
   \quad\raise-0.4in\hbox{\input bitr }%
   \vspace{-\baselineskip}
   \caption{The {\tt bi\_tr({\sl linespec},L|R,P,E)} macro.}
   \label{bitr}
   \end{figure}

The {\tt bi\_tr} and {\tt igbt} macros are wrappers for
the macro {\tt bi\_trans({\sl linespec}, L|R, {\sl chars}, E)}, which
draws the components of the transistor according to the characters in its
third argument.  For example, multiple emitters and collectors can be
specified as shown in Figure~\ref{bi_trans}.
\begin{figure}[h!]
\vspace*{-0.5\baselineskip}
   \input bi_trans
   \caption{The {\tt bi\_trans({\sl linespec},L|R,{\sl chars},E)} macro.
   The sub-elements are specified by the third argument.  The substring
   {\tt E}{\sl n} creates multiple emitters {\sl E0} to {\sl En}.
   Collectors are similar.}
   \label{bi_trans}
   \end{figure}

A UJT macro with predefined internal locations {\sl B1,} {\sl B2,}
and {\sl E} is shown in Figure~\ref{ujt},
and a thyristor macro with predefined internal locations
 {\sl G} and {\sl T1,} {\sl T2,} or
 {\sl A,} {\sl K} is in Figure~\ref{thyristor}.
\begin{figure}[h!]
\vspace*{-0.5\baselineskip}
   \input ujt
   \caption{UJT devices, with current drawing direction up. }
   \label{ujt}
   \end{figure}
\begin{figure}[h!]
\vspace*{-0.5\baselineskip}
   \input thyristor
   \vspace*{-1ex}
   \caption{The {\tt thyristor({\sl linespec, chars})} macro,
    drawing direction down. The element is not two-terminal, so the
    {\sl linespec} determines length and direction but not position.
    The {\tt scr} macro places the thyristor as a two-terminal element.}
   \label{thyristor}
   \end{figure}
Except for the {\sl G} terminal, a thyristor (the {\tt C} variant excluded)
is much like an two-terminal element.  The wrapper macro
{\tt scr({\sl linespec, chars, label})} draws a thyristor and places it
using {\sl linespec} as for a two-terminal element,
but requires a third argument for the label for the compound block; thus,

{\tt scr(from A to B,UA,Q3); line right from Q3.G}

\noindent
draws the element from position {\sl A} to position {\sl B} with label
{\sl Q3}, and draws a line from {\sl G}.

The number of possible semiconductor symbols is very
large, so these macros must be regarded as prototypes.
Often an element is a minor modification of existing elements.  For example,
the {\tt thyristor({\sl linespec}, {\sl chars})} macro illustrated in
Figure~\ref{thyristor} is derived from the diode and bipolar transistor macros.
Another example is the {\tt tgate} macro shown in Figure~\ref{Tgate}, which
also shows a pass transistor.
\enlargethispage{\baselineskip}%
\begin{figure}[h!t]
   \input Tgate
   \caption{The {\tt tgate({\sl linespec,} [B][R|L])} element, derived from
     a customized diode and {\tt ebox}, and the
     {\tt ptrans({\sl linespec}, [R|L])} macro.
     These are not two-terminal elements, so the {\sl linespec} argument
     defines the direction and length of the line from $A$ to $B$ but not
     the element position.}
   \label{Tgate}
   \end{figure}

Some FETs with predefined internal locations {\sl S,} {\sl D,} and {\sl G} are
also included, with similar arguments to those of {\tt bi\_tr,} as shown in
Figure~\ref{fet}.
In all cases the first argument is a linespec, and entering
{\tt R} as the second argument orients the {\sl G} terminal to the right of the
current drawing direction.
The macros in the top three rows of the figure are wrappers for the
general macro {\tt mosfet({\sl linespec},R,{\sl characters},E)}.
The third argument of this macro is a subset of the characters
$\{${\tt BDEFGLMQRSTXZ}$\}$, each letter corresponding to
a diagram component as shown in the bottom row of the figure. 
Preceding the characters {\tt B}, {\tt G}, and {\tt S} by {\tt u} or {\tt d}
adds an up or down arrowhead to the pin, preceding {\tt T} by {\tt d}
negates the pin, and preceding {\tt M} by {\tt u} or {\tt d} puts the pin
at the drain or source end respectively of the gate.
The obsolete letter {\tt L} is equivalent to {\tt dM} and has been kept
temporarily for compatibility.
This system allows considerable freedom in choosing or customizing components,
as illustrated in Figure~\ref{fet}.
\begin{figure}[ht]
   \input fet
   \caption{JFET, insulated-gate enhancement and depletion MOSFETs,
     and simplified versions.
     These macros are wrappers that invoke the {\tt mosfet}
     macro as shown in the bottom row.
     The two lower-right examples show custom devices, the first
     defined by omitting the substrate connection, and the second
     defined using a wrapper macro.}
   \label{fet}
   \end{figure}

Some other non-two-terminal macros are {\tt dot}, which has an
optional argument `{\tt at} {\sl location}', the line-thickness
macros, the {\tt fill\_} macro, and {\tt crossover}, which is a useful if
archaic method to show non-touching conductor crossovers, as in
Figure~\ref{bistable}.
\begin{figure}[h!t]
   \input bistable
   \vspace{-1ex}
   \caption{Bipolar transistor circuit, illustrating {\tt crossover}
      and colored elements.}
   \label{bistable}
   \end{figure}
This figure also illustrates now elements and labels can be colored
using the macro

{\tt rgbdraw({\sl r}, {\sl g}, {\sl b}, {\sl drawing commands})}

\noindent
where the {\sl r, g, b} values are in the range 0 to 1 to specify the rgb color.
This macro is a wrapper for the following, which may be more convenient
if many elements are to be given the same color:

   {\tt setrgb({\sl r}, {\sl g}, {\sl b})}
      \hfill\break\hspace*{\parindent}%
   {\sl drawing commands}
      \hfill\break\hspace*{\parindent}%
   {\tt resetrgb}

A macro is also provided for colored fills:

{\tt rgbfill({\sl r}, {\sl g}, {\sl b}, {\sl drawing commands})}

\noindent%
These macros depend heavily on the postprocessor and are intended only for 
\PSTricks, \TPGF, \MetaPost, and the Postscript output of \dpic.


\xection{Directions, looping, and corners\label{Directions}}
Aside from its block-structure capabilities, looping, and macros, 
\pic\ has a very useful concept of the current point and current direction,
the latter unfortunately limited to {\tt up, down, left, right.}
The circuit macros need to know the current direction, so
whenever {\tt up, down, left, right} are used they should be written
respectively as the macros {\tt up\_, down\_, left\_, right\_}.
To allow drawing circuit objects in other than the standard four directions,
a transformation matrix
is applied at the macro level to generate the required \pic\ code.
Potentially, the matrix can be used for other transformations.
The macros {\tt Point\_(}{\sl degrees}{\tt ),}
{\tt point\_(}{\sl radians}{\tt ),}
and {\tt rpoint\_(}{\sl rel linespec}{\tt )} re-define the entries
{\tt m4a\_}, {\tt m4b\_}, {\tt m4c\_}, {\tt m4d\_}
of the matrix for the required rotation.

The macro {\tt eleminit\_} in the two-terminal elements invokes
{\tt rpoint\_} with a specified or default {\sl linespec}
to establish element length and direction.
As shown in Figure~\ref{Oblique},
`{\tt Point\_(-30); resistor}' draws a resistor
along a line with slope of~-30 degrees, and `{\tt rpoint\_(to Z)}' sets
the current direction cosines to point from the current location to location Z.
Macro {\tt vec\_(x,y)}
evaluates to the position {\tt (x,y)} rotated as defined by the
argument of the previous {\tt Point\_, point\_} or {\tt rpoint\_} command.
The principal device used to define relative locations in the circuit macros
is {\tt rvec\_(x,y)}, which evaluates to position {\tt Here + vec\_(x,y)}.
Thus, {\tt line to rvec\_(x,0)} draws a line of length {\tt x} in the current
direction.
\begin{figure}[!ht]
\vspace{-\baselineskip}
   \parbox{4.5in}{\small \verbatiminput{Oblique.m4}}%
   \hfill\raise-0.7in\llap{\hbox{\input Oblique }}%
   \vspace{-\baselineskip}
   \caption{Illustrating elements drawn at oblique angles.}
   \label{Oblique}
   \end{figure}

Figure~\ref{Oblique} illustrates that some hand-placement of labels
using {\tt dlabel} may be useful when elements are drawn obliquely.
The figure also illustrates that any commas within \Mfour\ arguments must
be treated specially because the arguments are separated by commas.
Argument commas are protected either by parentheses as in
{\tt inductor(from Cr to Cr+vec\_(elen\_,0))}, or by multiple single quotes
as in {\tt `{}`,'{}',} as necessary.  Commas also may be avoided by writing
{\tt 0.5 between L and T} instead of {\tt 0.5<L,T>.}

Sequential actions can be performed using either the
 \dpic

{\tt for {\sl variable}={\sl expression} to {\sl expression}
 by {\sl expression} do $\lbrace$ {\sl actions} $\rbrace$}

command or at the
\Mfour\ processing stage.  The {\tt libgen} library defines the macro

{\tt for\_({\sl start}, {\sl end}, {\sl increment}, `{\sl actions}')}

\noindent
for this and other purposes.  Nested loops are allowed and the innermost loop
index variable is {\tt m4x.}
The first three arguments must be
integers and the {\sl end} value must be reached exactly; for example,
\verb|for_(1,3,2,`print In`'m4x')| prints locations {\sl In1} and {\sl In3},
but \verb|for_(1,4,2,`print In`'m4x')| does not terminate since the
index takes on values 1, 3, 5, $\ldots$.

Repetetive actions can also be performed with the the {\tt libgen} macro

{\tt Loopover\_(`{\sl variable}', {\sl actions}, {\sl value1}, {\sl value2},
  $\ldots$)}

\noindent
which evaluates {\sl actions} for each instance of {\sl variable} set
to {\sl value1, value2, $\ldots$}.

\pagebreak
If two straight lines meet at an angle, then depending on the postprocessor,
the corner may not be mitred or rounded unless the two lines belong to
a multisegment line, as illustrated in Figure~\ref{Corners}.  This is normally
not an issue for circuit diagrams unless the figure is magnified or thick
lines are drawn.  Rounded corners can be obtained by setting post-processor
parameters, but the figure shows the effect of
two macros {\tt corner} for right angles, and {\tt round},
that may assist in some cases. Otherwise, a two-segment line can be overlaid
at the corner to produce the same effect.
\begin{figure}[ht]
   \input Corners
   \caption{Producing mitred angles and corners.}
   \label{Corners}
   \end{figure}

\xection{Logic gates}
Figure~\ref{Logic} shows the basic logic gates included in
library {\tt liblog.m4}.
Gate macros have an optional argument, an integer $N$ from $0$ to 16,
defining locations {\tt In1,} $\ldots$ {\tt In}$N,$ as illustrated for
the NOR gate in the figure.  
By default $N=2,$ except for macros {\tt NOT\_gate} and {\tt
BUFFER\_gate}, which have one input {\tt In1} unless they are given a
first argument, which is treated as the line specification of a
two-terminal element.

Negated inputs or outputs are marked by circles drawn by the
\verb|NOT_circle| macro.  The name marks the point at the outer edge of the
circle and the circle itself has the same name prefixed by {\tt N\_}.
For example, the output circle of a nand gate is named
{\tt N\_Out} and the outermost point of the circle is named {\tt Out.}
The macro {\tt IOdefs} creates a sequence of named outputs.
\begin{figure}[t]
   \input Logic
   \caption{Basic logic gates.  The input and output locations of
      a three-input NOR gate are shown.  Inputs are negated by
      including an {\tt N} in the second argument letter sequence.  A {\tt B}
      in the second argument produces a box shape as shown in the rightmost
      column, where the second example has AND functionality and
      the bottom two are examples of exclusive OR functions.}
   \label{Logic}
   \end{figure}

Gates are typically not two-terminal elements and are normally drawn
horizontally or vertically (although arbitrary directions may be set
with e.g.\ {\tt Point\_({\sl degrees})}).
Each gate is contained in a
block of typical height {\tt 6*L\_unit} where {\tt L\_unit} is a macro
intended to establish line separation for an imaginary grid on which
the elements are superimposed.

Including an \verb|N| in the second
argument character sequence of any gate negates the inputs, and including
\verb|B| in the second argument invokes the
general macro {\tt BOX\_gate([P|N]...,[P|N],{\sl horiz size},{\sl
vert size},{\sl label})}, which draws box gates.  Thus, {\tt
BOX\_gate(PNP,N,,8,\char92 geq 1)} creates a gate of default width,
eight {\tt L\_unit}s height, negated output, three inputs with the
second negated, and internal label ``$\geq1$''.
If the fifth argument begins with {\tt sprintf} or a double quote then
the argument is copied literally; otherwise it is treated as scriptsize
mathematics.

Input locations retain their positions relative to the gate body
regardless of gate orientation, as shown in Figure~\ref{FF}.
Beyond a default number (6) of inputs, the
gates are given wings as illustrated in Figure~\ref{exVIII}.
\begin{figure}[h!t]
   \vspace*{-\baselineskip}
   \parbox{4.75in}{\small \verbatiminput{FF.m4}}%
   \quad\input FF
   \vspace*{-\baselineskip}
   \caption{$SR$ flip-flop.}
   \label{FF}
   \end{figure}
\begin{figure}[h!t]
   \input mplex
   \vspace*{-\baselineskip}
   \caption{Eight-input multiplexer with
    {\tt for\_} looping in the source, showing a gate with wings.}
   \label{exVIII}
   \end{figure}
\begin{figure}[h!t]
   \input FlipFlop
   \caption{The {\tt FlipFlop} and {\tt Mux} macros, with variations.}
   \label{FlipFlops}
   \end{figure}

Figure~\ref{FlipFlops} shows a multiplexer block with variations, and
the macro {\tt FlipFlop(D|T|RS|JK, {\sl label, boxspec})}, which is a
wrapper for the more specific {\tt FlipFlop6(}{\sl label, spec,
boxspec}{\tt )} and {\tt FlipFlopJK(}{\sl label, spec, boxspec}{\tt )}
macros.  Pins on the latter two can be omitted or negated according to
their second argument.  The second argument of {\tt FlipFlop6}, for
example, contains {\tt NQ, Q, CK, S, PR, CLR} to include these pins.
Preceding any of these with {\tt n} negates the pin.  The substring
{\tt lb} is included to write labels on the pins.  Any other substring
applies to the top left pin, with {\tt .} equating to a blank.  Thus,
the second argument can be used to customize the flip-flop.

A good strategy for drawing complex logic circuits might be summarized
as follows:
\begin{itemize}
\item Establish the absolute locations of gates and other major components
  (e.g.\ chips) relative to a grid of mesh size commensurate with
  {\tt L\_unit}, which is an absolute length.
\item Draw minor components or blocks relative to the major ones, using
   parametrized relative distances.
\item Draw connecting lines relative to the components and previously drawn
   lines.
\item Write macros for repeated objects.
\item Tune the diagram by making absolute locations relative, and by tuning
   the parameters.
   Some useful macros for this are the following, which are in units of
  {\tt L\_unit}:
   \begin{itemize}
   \item[] {\tt AND\_ht, AND\_wd}: the height and width of basic AND and
     OR gates
   \item[] {\tt BUF\_ht, BUF\_wd}: the height and width of basic buffers
   \item[] {\tt N\_diam}: the diameter of NOT circles
   \end{itemize}
   \end{itemize}
In addition to the logic gates described here, some experimental
IC chip diagrams are included with the distributed example files.

Customized gates can be defined simply.
For example, the following code defines the custom flip-flops in
Figure~\ref{ShiftR}.
\begin{verbatim}
define(`customFF', `[ Chip: box wid 10*L_unit ht FF_ht*L_unit
    ifelse(`$1',1,`lg_pin(Chip.se+svec_(0,int(FF_ht/4)),lg_bartxt(Q),PinNQ,e)')
    lg_pin(Chip.ne-svec_(0,int(FF_ht/4)),Q,PinQ,e)
    lg_pin(Chip.w,CK,PinCK,wEN)
    lg_pin(Chip.n,PR,PinPR,nN)
    lg_pin(Chip.s,CLR,PinCLR,sN)
    lg_pin(Chip.sw+svec_(0,int(FF_ht/4)),R,PinR,w)
    lg_pin(Chip.nw-svec_(0,int(FF_ht/4)),S,PinS,w) ]')
\end{verbatim}
This definition makes use of macros \verb|L_unit| and
\verb|FF_ht| that predifine dimensions and the
logic-pin macro \verb|lg_pin(|{\sl location, printed label, pin name,
type}\verb|)|.  The pin $\overline{\hbox{\sf Q}}$ is drawn only if the
macro argument is 1.
\begin{figure}[h!b]
   \input ShiftR
   \caption{A 5-bit shift register.}
   \label{ShiftR}
   \end{figure}

For hybrid applications, the \verb|dac| and \verb|adc| macros are
illustrated in Figure~\ref{Dac}.
The figure shows the default and predifined internal locations, the number
of which can be specified as macro arguments.
\begin{figure}[ht]
   \input Dac
   \caption{The {\tt dac({\sl width,height},nIn,nN,nOut,nS)}
   and {\tt adc({\sl width,height},nIn,nN,nOut,nS)} macros.}
   \label{Dac}
   \end{figure}

\iffalse
A good strategy for drawing complex logic circuits might be summarized
as follows:
\begin{itemize}
\item Establish the absolute locations of gates and other major components
  (e.g.\ chips) relative to a grid of mesh size commensurate with
  {\tt L\_unit}, which is an absolute length.
\item Draw minor components or blocks relative to the major ones, using
   parametrized relative distances.
\item Draw connecting lines relative to the components and previously drawn
   lines.
\item Write macros for repeated objects.
\item Tune the diagram by making absolute locations relative, and by tuning
   the parameters.
   Some useful macros for this are the following, which are in units of
  {\tt L\_unit}:
   \begin{itemize}
   \item[] {\tt AND\_ht, AND\_wd}: the height and width of basic AND and
     OR gates
   \item[] {\tt BUF\_ht, BUF\_wd}: the height and width of basic buffers
   \item[] {\tt N\_diam}: the diameter of NOT circles
   \end{itemize}
   \end{itemize}
In addition to the logic gates described here, some experimental
IC chip diagrams are included with the distributed example files.
\fi


\xection{Element and diagram scaling}\label{scaling:}

There are several issues related to scale changes.  You may wish to use
millimetres, for example, instead of the default inches.  You may wish
to change the size of a complete diagram while keeping the relative
proportions of objects within it.  You may wish to change the sizes or
proportions of individual elements within a diagram.  You must take
into account that line widths are scaled separately from drawn objects,
and that the size of typeset text is independent of the \pic\ language.

The scaling of circuit elements will be described first, then
the \pic\ scaling facilities.

\subsection{Circuit scaling}\label{Circuitscaling:}
The circuit elements all have default dimensions
that are multiples of the \pic\ environmental parameter {\tt linewid,}
so changing this parameter changes default element dimensions.
The scope of a \pic\ variable is the current block; therefore, a sequence
such as

\begin{verbatim}
  resistor
T: [linewid = linewid*1.5; up_; Q: bi_tr] with .Q.B at Here
  ground(at T.Q.E)
  resistor(up_ dimen_ from T.Q.C)
\end{verbatim}

\noindent%
connects two resistors and a ground to an enlarged transistor.
Alternatively, you may redefine the default length {\tt elen\_}
or the body-size parameter {\tt dimen\_.}  For example, adding the line

{\tt define(`dimen\_',dimen\_*1.2)}

\noindent%
after the {\tt cct\_init} line of {\tt quick.m4} produces slightly
larger body sizes for all circuit elements.

\subsection{Pic scaling}
There are at least three kinds of graphical elements to be considered:
\begin{enumerate}
\item When generating final output after reading the {\tt.PE} line,
  \pic\ processors divide distances and sizes by the value of the
  environmental parameter {\tt scale}, which is 1 by default.  Therefore,
  the effect of assigning a value to {\tt scale} at the beginning of the
  diagram is to change the drawing unit (initially 1 inch) throughout
  the figure.  For example, the file {\tt quick.m4} can be modified to
  use millimetres as follows:
  \begin{verbatim}
  .PS                            # Pic input begins with .PS
  scale = 25.4                   # mm
  cct_init                       # Set defaults

  elen = 19                      # Variables are allowed
  ...
  \end{verbatim}
\vspace*{-1.5\baselineskip}
  The default sizes of \pic\ objects
  are redefined by assigning new values to the environmental parameters
  {\tt arcrad,} {\tt arrowht,} {\tt arrowwid,} {\tt boxht,} {\tt boxrad,}
  {\tt boxwid,} {\tt circlerad,} {\tt dashwid,} {\tt ellipseht,}
  {\tt ellipsewid,} {\tt lineht,} {\tt linewid,} {\tt moveht,}
  {\tt movewid,}
  {\tt textht,} and {\tt textwid.}
  The $\ldots${\tt ht} and $\ldots${\tt wid} parameters refer to the
  default sizes of vertical and horizontal lines, moves, etc., except for
  {\tt arrowht} and {\tt arrowwid}, which are arrowhead dimensions.
  The {\tt boxrad} parameter can be used to put rounded corners on boxes.
  Assigning a new value to {\tt scale} also multiplies all of these 
  parameters except {\tt arrowht,} {\tt arrowwid,} {\tt textht,} and
  {\tt textwid} by the new value of {\tt scale} (\gpic\ multiplies them all).
  Therefore, objects drawn to default sizes are unaffected by changing
  {\tt scale} at the beginning of the diagram.
  To change default sizes, redefine the appropriate parameters explicitly.

\item The {\tt .PS} line can be used to scale the entire drawing, regardless
  of its interior.  Thus, for example, the line {\tt.PS 100/25.4}
  scales the entire drawing to a width of 100$\,$mm.
  Line thickness, text size, and \dpic\ arrowheads are unaffected by
  this scaling.

  If the final picture width exceeds {\tt maxpswid}, which
  has a default value of 8.5, then the picture is scaled to this size.
  Similarly, if the height exceeds {\tt maxpsht} (default 11), then the
  picture is scaled to fit.  These parameters can be assigned
  new values as necessary, for example, to accommodate landscape figures.

\item The finished size of typeset text is independent of \pic\ variables,
  but can be determined as in Section~\ref{Interaction:}.  Then,
  {\tt "text" wid $x$ ht $y$} tells \pic\ the size of {\tt text},
  once the printed width $x$ and height $y$ have been found.

\item Line widths are independent of diagram and text scaling, and have
  to be set explicitly.  For example,
  the assignment {\tt linethick = 1.2} sets the default line width to 1.2\,pt.
  The macro {\tt linethick\_({\sl points})} is also provided, together
  with default macros {\tt thicklines\_} and {\tt thinlines\_}.

\end{enumerate}

\xection{Writing macros}\label{Writing:}
The \Mfour\ language is quite simple and is
described in numerous documents such as the original
reference~\cite{KRm4} or in later manuals~\cite{Seindal94}.  If a new
element is required, then modifying and renaming one of
the library definitions or simply adding an option to it may
suffice.  Hints for drawing general two-terminal elements are given in
{\tt libcct.m4}.  However, if an element or block is to be drawn in
only one orientation then most of the elaborations used for general
two-terminal elements in Section~\ref{Basictwo} can be dropped.

It may not be necessary to define your own macro if all that is needed is
a small addition to an existing element that is defined in an enclosing
\verb|[ ]| block.  After the element arguments are expanded,
one argument beyond the normal list is automatically expanded before
exiting the block, as mentioned near the beginning of Section~\ref{Other}.
This extra argument can be used to embellish the element.  

A macro is defined using quoted name and replacement text as follows:

{\tt define(`}{\sl name}{\tt',`}{\sl replacement text}{\tt ')}

After this line is read by the \Mfour\ processor, then whenever {\sl name}
is encountered as a separate string, it is replaced by its replacement
text, which may have multiple lines.  The quotation characters are used
to defer macro expansion.  Macro arguments are referenced inside a
macro by number; thus {\tt \$1} refers to the first argument.
A few examples will be given.

\noindent\hbox{}\\ {\bf Example 1:}
Custom two-terminal elements can often be defined by writing a wrapper
for an existing element.  For example, an enclosed thermal switch
can be defined as shown in Figure~\ref{Thermal}.
\begin{figure}[hbt]
   \parbox{4in}{\tt define(`thermalsw',\hfill\break
   \hbox{}\space`dswitch(`\$1',`\$2',WDdBT)\hfill\break
   \hbox{}\space\space circle rad distance(M4T,last line.c) at last line.c') }%
   \hfill\raise-0.15in\hbox{\input Thermal }
   \caption{A custom thermal switch defined from the {\tt dswitch} macro.}
   \label{Thermal}
   \end{figure}

\noindent {\bf Example 2:}
In the following,
two macros are defined to simplify the repeated drawing
of a series resistor and series inductor, and the macro {\tt tsection} defines
a subcircuit that is replicated several times to generate Figure~\ref{Tline}.
{\small \verbatiminput{Tline.m4}}
\begin{figure}[hbt]
   \input Tline
   \caption{A lumped model of a transmission line, illustrating the
    use of custom macros.}
   \label{Tline}
   \end{figure}

%\pagebreak
\noindent {\bf Example 3:}
Repeated subcircuits might have different orientations.
Suppose that a simple opamp subcircuit might have to be drawn in any direction.
The subcircuit will be placed in a {\tt [} {\tt ]} block, with
internal points {\sl In}, {\sl Out}, and {\sl G}. 
The macro interface could be something like the following:

{\tt fbfilter( U|D|L|R|{\sl degrees}, [L|R], {\sl opamp label},
  {\sl C label}, {\sl R label} )}

\noindent The first argument specifies the drawing direction as for the
{\tt antenna} macro, for example.  Setting the second argument to {\tt
R} specifies right orientation with respect to the drawing direction,
and the last three arguments are labels
for three internal elements.  Two instances of this subcircuit are
drawn and placed by the following code, with the result
shown in Figure~\ref{fbfilter}.
\begin{verbatim}
F1: fbfilter(,,K_3,C_{24},R_4)
  ground(at F1.G)
  dot(at F1.In); line up_ elen_/4
F2: fbfilter(L,R,K_2,C_{23},R_3) with .In at F1.In
  ground(at F2.G)
\end{verbatim}
\begin{figure}[hbt]
   \input fbfilter
   \caption{Showing the result of two invocations of the {\tt fbfilter} macro,
     with labels.}
   \label{fbfilter}
   \end{figure}
A draft macro for the subcircuit follows:
\begin{verbatim}
define(`fbfilter',
`[ direction_(ifelse(`$1',,0,`$1')) # Process arg 1, default to the right
   eleminit_                        # Assign rp_ang, rp_len
   tmpang = rp_ang                  # Save rp_ang 
   hunit = elen_                    # Dimension parameters
   vunit = ifinstr(`$2',R,-)elen_/2
 K: opamp(,,,,`$2')
   move to K.In`'ifinstr(`$2',R,2,1); line to rvec_(-hunit/4,0)
 J: dot
 R: resistor(to rvec_(-elen_,0)); point_(tmpang) # Reset rp_ang
 In: Here
   move to K.In`'ifinstr(`$2',R,1,2); line to rvec_(-hunit/4,0)
 G: Here
   dot(at K.Out)
   { line to rvec_(hunit/4,0)
 Out: Here } 
   line to rvec_(0,vunit)
 C: capacitor(to rvec_(-distance(K.Out,0.5 between J and G),0)); point_(tmpang)
   line to J
   ifelse(`$3',,,"$`$3'$" at K.C)   # Add the labels if non-blank.
   ifelse(`$4',,,"$`$4'$" at C+vec_(0,-vunit/3))
   ifelse(`$5',,,"$`$5'$" at R+vec_(0,-vunit/4))
   ]')
\end{verbatim}
The drawing direction is unknown when the macro is defined, so the
macros {\tt vec\_} and {\tt rvec\_} are used for drawing lines and
elements.
Thus, {\tt (vec\_({\sl x,y}))} is position {\tt ({\sl x,y})} rotated
by angle {\tt rp\_ang}.
A side effect of drawing a two-terminal element is to change the
drawing direction (in conformity with the \pic\ language), so the angle must
be saved and reset as needed. Normally, the {\tt fbfilter} block will be
placed by specifying the position of one of its defined points; by
default it will be placed as if it were a box.

\xection{Interaction with \LaTeX}\label{Interaction:}
The sizes of typeset labels and other \TeX\ boxes are generally unknown
prior to processing the diagram by \LaTeX.
Although they are not needed for many circuit diagrams,
these sizes may be required explicitly for calculations or implicitly
for determining the diagram bounding box.  For example, the 
text sizes in the following example affect the total size of the diagram:

\begin{verbatim}
.PS
B: box
  "Left text" at B.w rjust
  "Right text: $x^2$" at B.e ljust
.PE
\end{verbatim}

The \pic\ interpreter cannot know the dimensions of the text to the left
and right of the box, and the diagram is generated using
default text dimensions.  One solution is to measure the text sizes by hand and
include them literally, thus:\hfill\break
\hbox{}\quad%
\verb|"Left text" wid 38.47pt__ ht 7pt__ at B.w rjust|\hfill\break
but this is tedious.

A better solution to this problem is to process the diagram twice.  The
diagram source is processed as usual by \Mfour\ and a \pic\ processor, and the
main document source is \LaTeX{}ed to input the diagram and format the
text, and also to write the required dimensions into a supplementary file.
Then the diagram source is processed again, reading the required
dimensions from the supplementary file and producing a diagram ready
for final \LaTeX{}ing.  This hackery is summarized below, with an example
in Figure~\ref{stringdim}.
\enlargethispage{\baselineskip}
\begin{itemize}
\item Put \verb|\usepackage{boxdims}| into the document source.
\item Insert the following at the beginning of the diagram source,
 where {\sl jobname} is the name of the main \LaTeX\ file:\hfill\break
 \quad{\tt sinclude({\sl jobname}.dim)\hfill\break
 \quad s\_init({\sl unique name})}
\item Use the macro {\tt s\_box({\sl text})} to produce
 typeset text of known size as shown in Figure~\ref{stringdim};
 alternatively, invoke the macros
 \verb|\boxdims| and \verb|boxdim| described later.
\end{itemize}
\begin{figure}[h!t]
   \parbox{3.5in}{\small\tt.PS\\
     sinclude(CMman.dim)\\
     s\_init(stringdims)\\
     B: box\\
       \hbox{}\quad s\_box(Left text) at B.w rjust\\
       \hbox{}\quad s\_box(Right text:\ \$x\^{}{\%g}\$,2) at B.e ljust\\
     .PE}%
   \hfill\llap{\raise-0.25in\hbox{\input stringdims }}%
   \caption{The macro {\tt s\_box} sets string dimensions automatically
    when processed twice.  If two or more arguments are
    given to {\tt s\_box}, they
    are passed through {\tt sprintf}.  The dots show the figure bounding box.}
   \label{stringdim}
   \end{figure}

\noindent The macro \verb|s_box(|{\sl text}\verb|)| evaluates initially to

 \verb|"\boxdims{|{\sl name}\verb|}{|{\sl text}\verb|}"|
  \verb|wid boxdim(|{\sl name}\verb|,w) ht boxdim(|{\sl name}\verb|,v)|

\noindent
On the second pass, this is equivalent to

 {\tt "{\sl text}" wid {\sl x} ht {\sl y}}

 \noindent
 where {\sl x} and {\sl y} are the typeset dimensions of the
 \LaTeX\ input text.  If {\tt s\_box} is given two or more arguments
 as in Figure~\ref{stringdim}
 then they are processed by {\tt sprintf}.

The argument of {\tt s\_init}, which should be unique within {\tt{\sl
jobname}.dim}, is used to generate a unique \verb|\boxdims| first
argument for each invocation of \verb|s_box| in the current file.  If
\verb|s_init| has been omitted, the symbols ``{\bf !!}'' are inserted
into the text as a warning.  Be sure to quote any commas in the
arguments.  Since the first argument of {\tt s\_box} is \LaTeX\ source,
make a rule of quoting it to avoid comma and name-clash problems.  For
convenience, the macros {\tt s\_ht}, {\tt s\_wd}, and {\tt s\_dp}
evaluate to the dimensions of the most recent {\tt s\_box} string or to
the dimensions of their argument names, if present.

The file \verb|boxdims.sty| distributed with this package should be installed
where \LaTeX\ can find it.
The essential idea is to define a two-argument \LaTeX\ macro
\verb|\boxdims| that writes out definitions for the width, height and
depth of its typeset second argument into file {\sl jobname.}\verb|dim|,
where {\sl jobname} is the name of the main source file.
The first argument of \verb|\boxdims| is used to construct unique symbolic
names for these dimensions. 
Thus, the line

{\tt box \verb|"\boxdims{Q}{\Huge Hi there!}"| }

\noindent has the same effect as

{\tt box \verb|"\Huge Hi there!"|}

\noindent except that the line

{\tt define(`Q\_w',77.6077pt\_\_)define(`Q\_h',17.27779pt\_\_)%
define(`Q\_d',0.0pt\_\_)dnl}

\noindent is written into file {\sl jobname.}\verb|dim|
(and the numerical values depend on the current font).
These definitions are required by the \verb|boxdim| macro
described below.

The \LaTeX\ macro

\verb|\boxdimfile{|{\sl dimension file}\verb|}|

\noindent is used to specify an alternative to {\sl jobname.}\verb|dim| as the
dimension file to be written.  This simplifies cases where {\sl jobname}
is not known in advance or where an absolute path name is required.

Another simplification is available.  Instead of the
{\tt sinclude({\sl dimension file})} line
above, the dimension file can be read by \Mfour\ before reprocessing the source
for the second time:

{\tt m4 {\sl library files} {\sl dimension file} {\sl diagram source file} ...} 

Objects can be tailored to their attached text by invoking
\verb|\boxdims| and \verb|boxdim| explicitly.
The small source file in Figure~\ref{boxdims}, for example,
produces the box in the figure.
\begin{figure}[hbt]
   \parbox{4.2in}{\small \input eboxdims.verb }%
   \hfill\llap{\raise-0.35in\hbox{\input eboxdims }}%
   \vspace{-\baselineskip}
   \caption{Fitting a box to typeset text.}
   \label{boxdims}
   \end{figure}

The figure is processed twice, as described previously.
The line \verb|sinclude(|{\sl jobname}\verb|.dim)| reads the named file
if it exists.  The macro \verb|boxdim(|{\sl
name,suffix,default}\verb|)| from {\tt libgen.m4} expands the
expression \verb|boxdim(Q,w)| to the value of \verb|Q_w| if it is
defined, else to its third argument if defined, else to 0, the latter
two cases applying if {\sl jobname.}\verb|dim| doesn't exist yet.  The
values of \verb|boxdim(Q,h)| and \verb|boxdim(Q,d)| are similarly
defined and, for convenience, \verb|boxdim(Q,v)| evaluates to the sum
of these.  Macro \verb|pt__| is defined as \verb|*scale/72.27| in {\tt
libgen.m4}, to convert points to drawing coordinates.

Sometimes a label needs a plain background in order to blank
out previously drawn components overlapped by the label,
as shown on the left of Figure~\ref{fbox}.
\begin{figure}[hbt]
   \input woodchips
   \vspace{-0.5\baselineskip}
   \caption{Illustrating the {\tt f\_box} macro.}
   \label{fbox}
   \end{figure}
The technique illustrated in Figure~\ref{boxdims} is automated by the
macro
{\tt f\_box(}{\sl boxspecs}, {\sl label arguments}{\tt )}.
For the special case of only one argument,
e.g., {\tt f\_box(Wood chips),} this macro
simply overwrites the label on a white box of identical size.
Otherwise, the first argument specifes the box characteristics
(except for size), and the macro evaluates to

{\tt box }{\sl boxspecs} {\tt s\_box(}{\sl label arguments}{\tt)}.
 
\noindent%
For example, the result of the following command
is shown on the right of Figure~\ref{fbox}.

\verb|f_box(color "lightgray" thickness 2 rad 2pt__,"\huge$n^{%g}$",4-1)|

More tricks can be played.  The example

\verb|Picture: s_box(`\includegraphics{|{\it file}\verb|.eps}') with .sw at|
{\sl location}

\noindent shows a nice way of including eps graphics in a diagram.  The
included picture (named {\tt Picture} in the example) has known position and
dimensions, which can be used to add vector graphics or text to the
picture.  To aid in overlaying objects, the macro {\tt boxcoord(}{\sl
object name, x-fraction, y-fraction}{\tt)} evaluates to a position,
with {\tt boxcoord(}{\sl object name}{\tt,0,0)} at the lower left
corner of the object, and {\tt boxcoord(}{\sl object name}{\tt,1,1)} at
its upper right.

\xection{\PSTricks\ and other tricks}
This section applies only to a \pic\ processor (\dpic) that is
capable of producing output compatible with
\PSTricks, \TPGF, or, in principle, other graphics postprocessors. 

By using {\tt command} lines,
or simply by inserting \LaTeX\ graphics directives along with strings to
be formatted, one can mix
arbitrary \PSTricks\ (or other) commands with \Mfour\ input
to create complicated effects. 

Some commonly required effects are particularly simple. For example,
the rotation of text by \PSTricks\ postprocessing is illustrated by the file

{\small \verbatiminput{Axes.m4}}

\noindent%
which contains both horizontal text and text rotated $90^\circ$ along the
vertical line.
This rotation of text is also implemented by the macro
{\tt rs\_box}, which is similar
to {\tt s\_box} but rotates its text argument by $90^\circ,$ a default angle
that can be changed by preceding invocation with
\verb|define(`text_ang',|{\sl degrees}{\tt )}.  The {\tt rs\_box} macro
requires either \PSTricks\ or \TPGF\ and, like {\tt s\_box}, it calculates the
size of the resulting text box but requires the diagram to be
processed twice.

Another common requirement is the filling of arbitrary shapes, as
illustrated by the following lines within a {\tt .m4} file:

\vspace{\parsep}
\noindent%
\verb|command "`\pscustom[fillstyle=solid,fillcolor=lightgray]{'"|
\hfill\break
{\sl drawing commands for an arbitrary closed curve}
\hfill\break
\verb|command "`}%'"|
\vspace{\parsep}

For colour printing or viewing, arbitrary
colours can be chosen, as described in the \PSTricks\ manual.
\PSTricks\ parameters can be set by inserting the line

\vspace{\parsep}
\noindent\verb|command "`\psset{|{\sl option=value,}$\;\ldots$\verb|}'"|
\vspace{\parsep}

\noindent%
in the drawing commands or by using the macro
{\tt psset\_(}{\sl PSTricks options}{\tt )}.

The macros
 {\tt shade(}{\sl gray value},{\sl closed line specs}{\tt )}
and
 {\tt rgbfill(}{\sl red value, green value, blue value, closed line specs}%
 {\tt )}
can be invoked to accomplish the same effect as the above fill example, but
are not confined to use only with \PSTricks.

Since arbitrary \LaTeX\ can be output, either in ordinary strings or by
use of {\tt command} output, complex examples such as found in
reference~\cite{Girou94}, for example, can be included.  The complications
are twofold: \LaTeX\ and \dpic\ may not know the dimensions of the formatted
result, and the code is generally unique to the postprocessor.
Where postprocessors are capable of equivalent results, then
macros such as {\tt rs\_box}, {\tt shade}, and {\tt rgbfill} mentioned
previously can be used to hide code differences.

\xection{Web documents, {\ttfamily pdf}, and alternative output formats}%
\label{Alternative:}

Circuit diagrams contain graphics and symbols, and the issues related to
web publishing are similar to those for other mathematical documents.
Here the important factor is that \gpic\ {\tt -t} generates output
containing \tpic\ \verb|\special| commands, which must be converted
to the desired output, whereas \dpic\ can generate several alternative
formats, as shown in Figure~\ref{Workflow}.  One of the easiest methods
for producing web documents is to generate postscript as usual and to
convert the result to pdf format with Adobe
Distiller\textregistered\ or equivalent.
\begin{figure}[h!t]
   \hskip-2em{ \input Workflow }
   \caption{Output formats produced by \gpic\ {\tt -t} and \dpic.
      SVG output can be read by Inkscape or used directly in web documents.}
   \label{Workflow}
   \end{figure}

PDFlatex produces pdf without first creating a postscript file
but does not handle \tpic\ \verb|\special|s, so \dpic\ must be
installed.

Most PDFLatex distributions are not directly compatible with \PSTricks, but
the \TPGF\ output of \dpic\ is compatible with both \LaTeX\ and PDFLatex.
Several alternative \dpic\ output formats such as
\mfpic\ and \MetaPost\ also work well.
To test \MetaPost, create a file {\sl filename}{\tt .mp}
containing appropriate header lines, for example:

\begin{verbatim}
  verbatimtex
  \documentclass[11pt]{article}
  \usepackage{times,boxdims,graphicx}
  \boxdimfile{tmp.dim}
  \begin{document} etex
\end{verbatim}
Then append one or more diagrams by using the equivalent of

{\tt m4 <}{\sl path}{\tt >mpost.m4 {\sl library files}
  {\sl diagram}.m4 | dpic -s >> {\sl filename}.mp}

The command ``{\tt mpost --tex=latex } {\sl filename}{\tt .mp end}''
processes this file, formatting the diagram text by creating a
temporary {\tt .tex} file, \LaTeX{}ing it, and recovering the {\tt .dvi}
output to create {\sl filename}{\tt .1} and other files.  If the {\tt boxdims}
macros are being invoked, this process must be repeated to handle
formatted text correctly as described in Section~\ref{Interaction:}.
In this case, either put {\tt sinclude(tmp.dim)} in the diagram {\tt .m4}
source or read the {\tt .dim} file at the second invocation of
\Mfour\ as follows:

{\tt m4 <}{\sl path}{\tt >mpost.m4 {\sl library files} tmp.dim
  {\sl diagram}.m4 | dpic -s >> {\sl filename}.mp}

On some operating systems, the absolute path name for {\tt tmp.dim} has
to be used to ensure that the correct dimension file is written and
read.  This distribution includes a {\tt Makefile} that simplifies the
process; otherwise a script can automate it.

Having produced {\sl filename}{\tt .1}, rename it to {\sl filename}{\tt .mps}
and, {\it voil\`a,} you can now run PDFlatex\ on a {\tt .tex} source
that includes the diagram using
\verb|\includegraphics{|{\sl filename}\verb|.mps}|
as usual.

The \dpic\ processor is capable of other output formats, as illustrated in
Figure~\ref{Workflow} and in example files included with the distribution.
The \LaTeX\ drawing commands alone or with {\tt eepic} or {\tt pict2e}
extensions are suitable only for simple diagrams. 

\xection{Developer's notes}
Years ago in the course of writing a book, I took a few days off to
write a \pic-like interpreter (\dpic) to automate the tedious
coordinate calculations required by \LaTeX\ picture objects.  The
macros in this distribution and the interpreter are the result of that
effort, drawings I have had to produce since, and suggestions received from
others.  The interpreter has been upgraded over time to generate
\mfpic, \MetaPost~\cite{metapost}, raw \Postscript, \Postscript with
{\tt psfrag} tags, and \PSTricks\ output, the latter my preference
because of its quality and flexibility, including facilities for colour
and rotations, together with simple font selection.
Ti{\it k}Z
PGF output, which combines the simplicity of \PSTricks\ with PDFlatex
compatibility, has been added.
\Xfig{}-compatible output was added early on to allow creating diagrams
by both by programming and interactive graphics.
Most recently, \SVG\ output has been added, and seems suitable for
producing web diagrams directly and for further editing by the
Inkscape interactive grapics editor.
Instead of \pic\ macros, I preferred the equally simple
but more powerful \Mfour\ macro processor, and therefore \Mfour\ is
required here, although \dpic\  now supports \pic-like macros.  Free
versions of \Mfour\ are available for Unix, Windows, and other
operating systems.

If starting over today would I not just use one of the other drawing
packages available these days?  It would depend on the context, but
\pic\ remains a good choice for line drawings since it is easy to learn
and read but powerful enough for coding the geometrical calculations
required for precise component sizing and placement. However, the main
value of this distribution is
not in the use of a specific language but in the element data encoded
in the macros, which have been developed and
refined over nearly two decades.  Some of them have become less readable
as more options and flexibility have been added, and if starting over
today, perhaps I would change some details.  No choice of tool is without
compromise, and producing good graphics seems to be time-consuming, no matter
how it is done.

The \dpic\ interpreter has several output-format options that may be
useful.  The {\tt eepicemu} and {\tt pict2e} extensions of the
primitive \LaTeX\ picture objects are supported.  The \mfpic\ output
allows the production of Metafont alphabets of circuit elements or
other graphics, thereby essentially removing dependence on device
drivers, but with the complication of treating every alphabetic
component as a \TeX\ box.  The \xfig\ output allows elements to be
precisely defined with \dpic\  and interactively placed with \xfig.
Similarly, the SVG output can be read directly by the Inkscape graphics
editor, but SVG can also be used directly for web pages.
\Dpic\ will also generate low-level \MetaPost\ or \Postscript\ code, so
that diagrams defined using \pic\ can be manipulated and combined with
others.  The \Postscript\ output is compatible with
CorelDraw\textregistered, and by extension to Adobe
Illustrator\textregistered.  With raw \Postscript\ output the user is
responsible for ensuring that the correct fonts are provided and for
formatting labels.

Many thanks to the people who continue to send comments, questions,
and, occasionally, bug fixes. What began as a tool for my own use grew
into a hobby that has persisted, thanks to your help and advice.

\xection{Bugs}\label{Bugs:}
The distributed macros are not written for maximum robustness.
Arguments could be entered in a key--value style (for example, {\tt
resistor(up\_ elen\_,style=N;cycles=8}) instead of by positional
parameters.  Macro arguments could be tested for correctness and
explanatory error messages could be written as necessary, but that
would make the macros more difficult to read and to write.  You will
have to read them when unexpected results are obtained or when you wish
to modify them.

In response to suggestions, some of the macros have been modified to
allow easier customization to forms not originally anticipated, but
these modifications are not complete.

Here are some hints, gleaned from experience and from comments I have
received.
\begin{enumerate}

\item {\bf Misconfiguration:} One of the following configuration files
  {\em must} be read by \Mfour\ before any of the other files, depending on the
  required form of \pic\ output:
    {\tt gpic.m4, pstricks.m4,
     postscript.m4,
     pgf.m4,
     mpost.m4, mfpic.m4, svg.m4,}
    or {\tt xfig.m4.}
  The package default is to read {\tt gpic.m4.}
  The processor options must be set correspondly,
  {\tt gpic -t} for {\tt gpic.m4} and, most often,
  {\tt dpic -p} or {\tt dpic -g} when \dpic\ is employed.
  For example, the pipeline for \PSTricks\ output from file {\tt circuit.m4} is

  {\tt m4 <path>pstricks.m4 <path>libcct.m4 circuit.m4 | dpic -p > circuit.tex}

  \noindent%
  but for \TPGF\ processing, the configuration file and \dpic\ option have to
  be changed:
  
  {\tt m4 <path>pgf.m4 <path>libcct.m4 circuit.m4 | dpic -g > circuit.tex}

  Any non-default configuration file must
  be processed explicitly as illustrated above.
  To redefine the default behaviour, change the {\tt include}
  statements near the top of the libraries.  The top-level makefile
  automates these changes.

\item 
{\bf Initialization:}
If the first element macro evaluated is not two-terminal or is within a
\Pic\ block, then later macros evaluated outside the block may produce
the error message

{\tt there is no variable `rp\_ang'}

\noindent because {\tt rp\_ang} is not defined in the outermost scope of the
diagram.  To cure this problem, make sure that the line

{\tt cct\_init}

\noindent appears immediately after the .PS line or prior to the first block.
It is entirely permissible to modify {\tt cct\_init} to include customized
diagram initializations such as the {\tt thicklines\_} statement.
For completeness, macros {\tt gen\_init,} {\tt log\_init,} {\tt darrow\_init}
are also provided for cases where the circuit library is not needed.

\item 
{\bf Pic objects versus macros:}  A common error is to write something like

{\tt line from A to B; resistor from B to C}

\noindent when it should be

{\tt line from A to B; resistor(from B to C)}

\noindent This error is caused by an unfortunate inconsistency between the
linear \pic\ objects and the way \Mfour\ passes macro arguments.

\item 
{\bf Commas:}
Macro arguments are separated by commas, so any comma that is
part of an argument must be protected by parentheses or quotes.  Thus,

{\tt shadebox(box with .n at w,h)}

\noindent produces an error, whereas

{\tt shadebox(box with .n at w`,'h)}

\noindent and

{\tt shadebox(box with .n at (w,h))}

\noindent do not.

\item 
{\bf Default directions and lengths:}
The {\em linespec} argument of element
macros requires both a direction and a length, and if either is omitted,
a default value is used.  Writing

{\tt source(up\_)}

\noindent draws a source up a distance equal to the current
{\tt lineht} value, which may cause confusion.
Writing

{\tt source(0.5)}

\noindent draws a source of length 0.5 units
in the current \pic\ default direction, which is one of
{\tt right,} {\tt left,} {\tt up,} or {\tt down.}
It is usually better
to specify both the direction and length of an element, thus:

{\tt source(up\_ elen\_).}

The effect of a {\em linespec} argument is independent of any direction
set using the {\tt Point\_} or similar macros. 
To draw an element at an obtuse angle (see Section~\ref{Directions}) try,
for example,

{\tt Point\_(45); source(to rvec\_(0.5,0))}

\item 
{\bf Quotes:} Single quote characters are stripped in pairs by \Mfour,
so the string

{\tt "`{}`inverse'{}'"}

\noindent will be typeset as if it were

{\tt "`inverse'".}

\noindent The cure is to add single quotes.

  The only subtlety required in writing
  \Mfour\ macros is deciding when to quote arguments.  In the context
  of circuits it seemed best to assume that macro arguments would not
  be protected by quotes at the level of macro invocation, but should
  be quoted inside each macro.  There may be cases where this rule is
  not optimal or where the quotes could be omitted.

\item 
{\bf Dollar signs:}
The $i$-th argument of an \Mfour\ macro is {\tt \$}$i,$ where $i$ is
an integer, so the following construction can cause an error when it
is part of a macro,

{\tt "\$0\$" rjust below}

\noindent since {\tt \$0} expands to the name of the macro itself.
To avoid this problem, put the string in quotes or write {\tt "\$`'0\$".}

\item 
{\bf Name conflicts:}  Using the name of a macro as part of
a comment or string is a simple and common error.
Thus,

{\tt arrow right \verb|"$\dot x$"| above}

\noindent produces an error message because {\tt dot} is a macro
name.   Macro expansion can be avoided by adding quotes, as follows:

{\tt arrow right `\verb|"$\dot x$"|' above}

Library macros intended only for internal use have names that begin
with {\tt m4} to avoid name clashes, but in addition, a good rule is to
quote all \LaTeX\ in the diagram input.

If extensive use of strings
that conflict with macro names is required, then one possibility is
to replace the strings by macros to be expanded by \LaTeX, for example
the diagram

{\tt
.PS

   \hspace*{\parindent} box \verb|"\stringA"|

.PE
}

\noindent with the LaTeX macro

{\tt
  \verb|\newcommand{\stringA}{|%

   \verb|Circuit containing planar inductor and capacitor}|
}

\item 
{\bf Current direction:} Some macros, particularly those for labels, do
unexpected things if care is not taken to preset the current direction
using macros {\tt right\_,} {\tt left\_,} {\tt up\_,} {\tt down\_,}
or {\tt rpoint\_($\cdot$).}
Thus for two-terminal macros it is good practice to write, e.g.

{\tt resistor(up\_ from A to B); rlabel(,R\_1)}

\noindent%
rather than 

{\tt resistor(from A to B); rlabel(,R\_1),}

\noindent%
which produce different results if the last-defined drawing direction is not
{\tt up}.  It might be possible to change the label macros to avoid this problem
without sacrificing ease of use.

\item {\bf Position of elements that are not 2-terminal:}
  The {\sl linespec} argument of elements defined in {\tt[} {\tt]}
  blocks must be understood as defining a direction and length, but
  not the position of the resulting block.
  In the \pic\ language, objects inside these brackets are placed by
  default {\em as if the block were a box}.  Place the
  element by its compass corners or defined interior points
  as described in the first paragraph of Section~\ref{Other} on
  page~\pageref{Other}, for example  
  
{\tt igbt(up\_ elen\_) with .E at (1,0)}

\item {\bf Pic error messages:} Some errors are detected only after scanning
  beyond the end of the line containing the error.  The semicolon
  is a logical line end, so putting a semicolon at the end of lines may
  assist in locating bugs. 

\item {\bf Scaling:} \Pic\ and these macros provide several ways to scale
  diagrams and elements within them, but subtle unanticipated effects
  may appear.  The line {\tt.PS} $x$ provides a convenient way to force
  the finished diagram to width $x.$  However if \gpic\ is the
  \pic\ processor then all scaled parameters are affected, including those
  for arrowheads and text parameters, which may not be the desired
  result.  A good general rule is to use the {\tt scale} parameter for
  global scaling unless the primary objective is to specify overall
  dimensions.

\item {\bf Buffer overflow:} For some \Mfour\ implementations,
  the error message {\tt pushed back more than 4096 chars}
  results from expanding large macros or macro arguments, and can be
  avoided by enlarging the buffer.  For example, the option {\tt
  -B16000} enlarges the buffer size to 16000 bytes.  However, this
  error message could also result from a syntax error.

\item {\bf \PSTricks\ anomaly:} If you are using \PSTricks\ and
  you get the error message {\tt Graphics parameter `noCurrentPoint'
  not defined..} then your version of \PSTricks\ is older than
  August 2010.  You can try the following:
  \begin{enumerate}
  \item Update your \PSTricks\ package.
  \item Instead, insert {\tt define(`M4PatchPSTricks',)} immediately after the
   {\tt .PS} line of your diagram.
  This change prevents the line
  \verb|\psset{noCurrentPoint}| from being added to the {\tt .tex}
  code for the diagram.  This line is a workaround for a ``feature''
  of the current \PSTricks\ \verb|\psbezier| command that changes its
  behaviour within the \verb|\pscustom| environment.  This situation
  occurs rarely and so the line is unnecessary for most diagrams.
  \item Comment out the second definition of {\tt M4PatchPSTricks} in
  {\tt pstricks.m4}.  The first definition works for some older
  \PSTricks\ distributions.
  To disable the workaround totally, change the second definition
  to be {\tt define(`M4PatchPSTricks',)}.
  \end{enumerate}


\end{enumerate}

\xection{List of macros}
\label{defines}
The following table lists the macros in libraries
darrow.m4, libcct.m4, liblog.m4, libgen.m4, and files gpic.m4, mfpic.m4,
and pstricks.m4.  Some of the sources in the {\tt examples}
directory contain additional macros, such as for flowcharts,
Boolean logic, and binary trees.

Internal macros defined within the libraries begin with the characters
m4 or M4 and, for the most part, are not listed here.

The library in which each macro is found is given, and a brief description.
\input defines
\endinput
